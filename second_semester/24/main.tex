\documentclass{article}

\usepackage[T2A]{fontenc}
\usepackage[utf8]{inputenc}
\usepackage[russian]{babel}
\usepackage{commath}
\usepackage{amsmath}
\usepackage{amsfonts}
\usepackage{amssymb} 
\usepackage{parskip}
\usepackage{titling}
\usepackage{color}
\usepackage{hyperref}
\usepackage[a4paper, left=2.5cm, right=1.5cm, top=2.5cm, bottom=2.5cm]{geometry}

\setlength{\droptitle}{-3cm}
\hypersetup{
    colorlinks=true, %set true if you want colored links
    linktoc=all,     %set to all if you want both sections and subsections linked
    linkcolor=blue,  %choose some color if you want links to stand out
}

\pagenumbering{arabic}

\begin{document}
    %RETHINK ALL "ARRAYS"
    \title{Числовые последовательности и их пределы}
    \author{Ученики 10-4 класса Оконешников Д.Д. и Паньков М.А. по лекции к.ф.-м.н. Протопоповой Т.В.}
    \date{от 21 апреля 2021 г.}
    \maketitle

    \section{Лекция №24}
    \subsection{Свойства бесконечно больших}

    \begin{enumerate}
        \item Если предел последовательности \( \lim_{n \rightarrow \infty }a_n = + \infty \)\( (+\infty + c) \), а \( \{b_n\} \) ограничена снизу, т.е. \( b_n \geq b\ \forall n \), тогда \(\lim_{n \rightarrow \infty}{(a_n + b_n)} = +\infty\)
       
        \item Если \( \lim_{n \rightarrow \infty }a_n = + \infty \)\( (+\infty + c) \), а \( \{b_n\} \) ограничена \( M: b_n \geq M > 0,\ \forall n \)
    \(\lim_{n \rightarrow \infty}{(a_n * b_n)} = +\infty\)

        \item Если \( \lim_{n \rightarrow \infty }a_n = +\infty \), а \( b_n \) ограничена, т.е. \( 0 < b_n < M (n \rightarrow \infty)\ \forall n \), то \[ (\frac{+\infty}{c > 0}) \lim_{n \rightarrow \infty}{\frac{a_n}{b_n}} = +\infty\]
    
        \item Если \(\lim_{n \rightarrow \infty}{a_n} = \infty\), а \(b_n\) ограничена; \(\abs{b_n} \leq M\  \forall n\), \[ (\frac{M}{\infty}) \lim_{n \rightarrow \infty }\frac{b_n}{a_n} = 0 \].
    \end{enumerate}

    \subsubsection{Неопределённости}

    \textbf{1)} \(\infty - \infty\)
    \\ \(2n - n = n \rightarrow +\infty\)
    \\ \(\lim_{n \rightarrow \infty}{\sqrt{n+1} - \sqrt{n}} = \lim_{n \rightarrow \infty}{\frac{n+1-n}{\sqrt{n+1}+\sqrt{n}}}\) = 0

    \textbf{2)} \( \frac{\infty}{\infty} \)

    \( n^2,\ n,\ 2n \)
    
    \( \frac{n^2}{n}  \) % TODO !!!!!!!!

    \textbf{3)} \(\infty * 0\)
    
    \subsection{Теорема Вейерштрасса}

    \textbf{Определение.}
    Числовая последовательность целых чисел называется стабилизирующейся к \(\xi\), если \( \exists\ n_0\ \forall n > n_0\  a_n = \xi: a_n \rightarrow \xi \)
    
    \textbf{Лемма 1.}
    Если \(\{a_n\}\) --- последовательность целых неотрицательных чисел, неубывающая и ограниченная сверху, т.е. \( a_n \leq N\ \forall n\), то \( \exists\ \xi: a_n \rightarrow \xi \) и \( \xi \leq M \).

    % IMAGE TODO!!!!
    % FOLLOWING TEXT RIGHT TO IMAGE
    хотя число членов последовательности \(\infty\), но между \(a_1\) (самый маленький член последовательности т.к. \( a_n \not\searrow \)) и \(M\) есть только конечное число целых чисел, \(\Rightarrow\) только конечно число значений \(a_n\)\\
    Обозначним наибольшее значение принимаемое \( a_n \), ч/з \( \xi \), т.е. \( \exists n_0: a_{n_0} = \xi \leq M \), тогда \( \forall n > n_0\ a_n = \xi \), т.к. \( a_n \downarrow \)
    
    \(\{a_n\}\) --- б.д.д. > 0
    
    \(a_1 = \alpha_{10},\alpha_{11}\alpha_{12}\alpha_{13}...\)\\
    \(a_2 = \alpha_{20},\alpha_{21}\alpha_{22}\alpha_{23}...\)\\
    \(a_3 = \alpha_{30},\alpha_{31}\alpha_{32}\alpha_{33}...\)\\
    ...\\
    \(a_n = \alpha_{n_0},\alpha_{n_1}\alpha_{n_2}\alpha_{n_3}...\)\\
    \(\downarrow a = \gamma_0,\gamma_1\gamma_2\gamma_3...\)

    \(\alpha_{n_0}\) --- целые неотр.\\
    \(\alpha_{n_j}\  j = 1, ...\) --- это \(\in \{0,\ 1,\ 2, \ ..., \ 9\}\)

    \textbf{Определение.} Будем говорить, что последовательность б.д.д. \( (>0) \{a_n\} \rightrightarrows \)\\
    \( a = \gamma_0,\gamma_1\gamma_2\gamma_3...\ (a_n \rightarrow a) \), если \(\forall k \ \alpha_{n_k} \rightrightarrows \gamma_k\)

    \textbf{Лемма 2.} Если \( \{a_n\} \) --- последовательность неотрицательных б.д.д. (*) является неубывающей и ограниченной (т.е. \( \exists M \) (б.д.д., не оканчивающася последовательностью 9-ок)): \( \forall\ n\ a_n \leq M) \), то \( \exists\ a \):
    
    1) \(a_n \rightrightarrows\ a\)  

    2) \(a_n \leq a \leq M\)

    \( \uparrow \) В табл. (*) cмотрим на первый столбец

    \( \alpha_{10} \)\\
    \( \alpha_{20} \)\\
    \( \alpha_{30} \)\\
    ...\\
    \( \alpha_{n0} \)

    Это последовательность неубывающих целых неотр. чисел и ограниченных сверху \(M\) по \textbf{Л1}\\
    \(\exists \) номер \(N_0\) \: \(\forall\ n > N_0\)\\
    \(\ \alpha_{n_0} \rightrightarrows\ \gamma_0\)
    
    \( \alpha_{10} \)\\
    \( \alpha_{20} \)\\
    \( \alpha_{30} \)\\
    ...\\
    \( \alpha_{N_00} = \gamma_0 \)\\
    \(\gamma_0\)\\
    \(\gamma_0\)

    Пусть \(n > N_1\), тогда смотрим на \(\{\alpha_{n1}\}\)

    \(\alpha_{n1}\) --- последовательность целых, неотр. чисел. Она ограничена 9-кой; неубывающая(т.к. \(a_n \not\searrow\) и 0-й столбец уже застабилиз.) \(\Rightarrow \ \exists\ N_1 \ \alpha_{n_1} \rightrightarrows\ \gamma_1 \ \forall n > N_1 \geq N_0\)

    Пусть \( n > N_1 \geq N_0 \) и смотрим \( \{ \alpha_{n2} \} \)
    
    \(\{\alpha_{n2}\}\) --- последовательность целых, неотр. чисел. Она ограничена 9-кой, неубывающей(т.к. \(a_n \not\searrow\) и 1-ый столбец уже застабилиз.) \(\Rightarrow \ \exists N_2 \ \: \ \forall n > N_2 \geq N_1 \geq N_0 \ \alpha_{n2} \rightrightarrows\ \gamma_2\) и т.д.

    в итоге \( \forall\ n > N_k \geq N_{k-1} \geq ... \geq N_0\quad \{ \alpha_{nk} \} \rightrightarrows\ \gamma_k \), то \( a_n \rightrightarrows\ a = \gamma_0,\gamma_1\gamma_2\gamma_3...\gamma_k... \)
    
    Из построения \(a_n \leq a \)
    \\ Осталось показать, что a \(\leq\) M

    Будем доказывать от противного: т.е. пусть \(a > M\), т.е. \(a_{(k)} = \gamma_0,\ \gamma_1,\ \gamma_2,\ ...,\ \gamma_k > M\)
    \\ \(a_k\) --- прибл. по недост. для a, но тогда \(a_n \ \: \ n > N_k\)
    \\ \(a_n = \alpha_{n0},\ \alpha_{n1},\ \alpha_{n2},\ ...,\ \alpha_{nk},\ \alpha_{nk+1},\ ... = \gamma_0,\ \gamma_1,\ ...,\ \gamma_k\alpha_{nk+1} > a_{(k)} > M\), противоречие с тем, что \(a_n \leq M \ \forall n \ \downarrow\)

    \subsubsection{Теорема Вейерштрасса}

    Если \(\{ x_n \} \) --- числовая последовательность неубывает и ограничена сверху, то она сходится.

    \(\uparrow\)
    \begin{enumerate}
        \item Пусть \( x_1 > 0 \Rightarrow \forall\ n\ x_n > 0 \) т.к. \( (x_n \not\searrow ) \).

        \item Любое \( x_n \in \mathbb{R} \) представлена в виде б.д.д.

        \item По \textbf{Л2}, такая (1) \( x_n \rightrightarrows a \)

        \item Покажем, что \(x_n \longrightarrow_{n \rightarrow \infty} a\)
    \\ Надо \(\forall \varepsilon > 0 \ \exists N \ \: \ \forall n > N\), \(\abs{x_n - a} < \varepsilon\)

    пусть \( n > N_k \), где \( N_k \) - номер, когда \(k\)-ый столбец в (*) застабилиз., тогда 
    \[ \abs{x_n - a} = \abs{\gamma_0,\gamma_1....\gamma_k\alpha_{nk+1}\alpha_{nk+2}... - \gamma_0,\gamma_1\gamma_2...\gamma_k\gamma{k+1}\gamma_{k+2}...} \]
    \[ = 0,\underbrace{0...0}_k\beta_{k+1}\beta_{k+2}... < 0,\underbrace{0...0}_{k-1}1 = \frac{1}{10^k} < \frac{1}{9k} < \varepsilon \]
    \[k > \frac{1}{9\varepsilon}\quad \varepsilon \rightarrow k \rightarrow N_k = N \downarrow \]
    \end{enumerate}

    \textbf{Замечание 1.} 
    Если \(x_1 < 0\), тогда рассм. \(y_1 = x_1 + c\ : \ y_1 > 0\), тогда по доказ. \( y_n = x_n + c \not\searrow \), ограничена сверху и \( y_1 > 0 \Rightarrow \) по доказ \(y_n \xrightarrow[n \rightarrow \infty]{}\ \Rightarrow ............. \)
    
    \textbf{Замечание 2.}
    Аналогично можно доказать, что, если \(\{x_n\} \not\nearrow\) и ограничена снизу, то она сх-ся.
    
    В общем случае:
    \\ Если последовательность монотонна и ограничена, то она сх-ся \(\downarrow\)
    
    \textbf{Пример.}

    \(a_{n+1} = \frac{a_n + 1}{2}\); \(a_1 = 2\), доказать, что \(\{a_n\}\) сх-ся и найти \(lim\).

    \(a_1 = 2;\ a_2 = \frac{1}{2}(2+1) = \frac{3}{2};\ a_3 = \frac{5}{4};\ a_4 = \frac{9}{8}\)
    \\ Предположение: \(\searrow\) и \(1 < a_n \leq 2\)

    \(a = \frac{a+1}{2} \\ 2a = a + 1 \\ a = 1\)

    I. Покажем, что \(1 < a_n \leq 2\)
    \ \ \ \ 1) База \(n = 1 \quad 1 < a_1 = 2 \leq 2\) верно

    \ \ \ \ 2) Пусть при \(n = k \quad 1 < a_k \leq 2\) выполнено

    \ \ \ \ 3) Надо при \(n = k + 1 \quad 1 < a_{k+1} \leq 2\)
    \\ \(1 = \frac{1}{2}(1+1) < \textrm{ (по ПИ) } a_{k+1} = \frac{a_k + 1}{2} = \frac{1}{2}(a_k + 1) \leq \frac{1}{2}(2 + 1) = \frac{3}{2} \leq 2\)
    \\ \(\Downarrow\) по ПМИ \(\forall n \ 1 < a_n \leq 2\)

    II. \( a_{n+1} - a_n \) = \( \frac{(a_n+1)}{2} - a_n = \frac{1-a_n}{2} (a_n > 1\) из опр \(<0) < 0 \Rightarrow a_{n+1} < a_n \forall\ n \in N \) т.е. \( a_n \) - убыв.

    III. Из I и II по Т. Вейерштрасса \( \exists\ lim_{n \rightarrow \infty}a_n \). Пусть \(lim_{n \rightarrow \infty } = a \Rightarrow a = \frac{a + 1}{2} \Rightarrow a = 1 \)


\end{document}
