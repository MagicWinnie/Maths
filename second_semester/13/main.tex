\documentclass{article}

\usepackage[T2A]{fontenc}
\usepackage[utf8]{inputenc}
\usepackage[russian]{babel}
\usepackage{commath}
\usepackage{amsmath}
\usepackage{amsfonts}
\usepackage{parskip}
\usepackage{titling}
\usepackage{color}
\usepackage{hyperref}
\usepackage[a4paper, left=2.5cm, right=1.5cm, top=2.5cm, bottom=2.5cm]{geometry}

\setlength{\droptitle}{-3cm}
\hypersetup{
    colorlinks=true, %set true if you want colored links
    linktoc=all,     %set to all if you want both sections and subsections linked
    linkcolor=blue,  %choose some color if you want links to stand out
}

\pagenumbering{arabic}

\begin{document}
    % \tableofcontents
    % \thispagestyle{empty}
    % \setcounter{tocdepth}{5}
    % \newpage

    % \addtocontents{toc}{\protect\contentsline{section}{\protect\numberline{}Элементы теории чисел}{}{}}
    \title{Элементы теории чисел. Теория сравнений.}
    \author{Ученик 10-4 класса Оконешников Д.Д. по лекции к.ф.-м.н. Протопоповой Т.В.}
    \date{от 20 января 2021 г.}
    \maketitle

    \section{Лекция №13}
    \subsection{Каноническое разложение числа. НОД. НОК}
    
    \begin{tabular}{ll}
        Весь алгоритм: & \textbf{Пример.} НОД(5083,3553)-?\\
        \textbf{1)} \( a = q_1b + r_1 \) & \( \Rightarrow r_1 = a - q_1b = A_1a + B_1b \)\\
        \textbf{2)} \( b = q_2r_1 + r_2 \) & \( \Rightarrow r_2 = b - q_2r_1 = b - q_2(A_1a + B_1b) = -q_2A_1a + (1 - B_1q_2)b = A_2a + B_2b \)\\
        \textbf{3)} \( r_1 = q_3r_2 + r_3 \) & \( \Rightarrow r_3 = r_1 - q_3r_2 = A_1a + B_1b - q_3(A_2a + B_2b) =  \)\\
        ...& \( = (A_1 - q_3A_2)a + (B_1 - q_3B_2)b = A_3a + B_3b \)\\
        \textbf{k)} \( r_{k-2} = q_kr_{k-1} + r_k \) & \( r_k = A_ka + B_kb \textrm{ или НОД}(a,b)=Aa + Bb \), где \( A,\ B \) --- целые\\
        \textbf{k+1)} \( r_{k-1} = q_{k+1}r_{k} + 0 \) & \\
        \( \textrm{НОД}(a,b) = r_k \) & \\
    \end{tabular}

    \textbf{Утверждение.} Если \( d = \textrm{ НОД}(a,b) \), то существуют целые \( A \) и \( B: d = Aa + Bb \).

    \textbf{Замечание.} Если \( \textrm{НОД}(a,b) = 1 \) (т.е. \( a \) и \( b \) взаимно просты), то существуют целые \( A \) и \( B: 1 = Aa + Bb \).

    \subsection{Доказательство свойств делимости 8 и 9}
    \textbf{Свойство 8.} Если \( ab \vdots m \) и \( \textrm{НОД}(a, m) = 1 \), то \( b \vdots m \)

    \( \uparrow \) Имеем \( \textrm{НОД}(a,m) = 1 \Rightarrow \exists A,\ M: Aa + Mm = 1. \)\\
    Домножим последнее равенство на \( b: \underset{\vdots m}{Aab} + \underset{\vdots m}{Mmb} = b \Rightarrow b \vdots m \downarrow \)

    \textbf{Свойство 9.} Если \( a \vdots m,\ a \vdots k \) и \( \textrm{НОД}(m,k) = 1 \), то \(a \vdots mk \)

    \( \uparrow \)

    1) \( a \vdots m \Rightarrow a = mq_1 \)\\
    2) \( a \vdots k \Rightarrow mq_1 \vdots k \)\\
    3) из 2) и \( \textrm{НОД}(m,k) = 1 \Rightarrow \) по свойству 8 \( q_1 \vdots k \Rightarrow q_1 = kq_2 \)\\
    4) \( a = mq_1 = mkq_2 \), т.е. \( a \vdots mk \downarrow \)

    \subsection{Решение уравнений ax + by = c}
    \textbf{Определение.} Диофантово уравнение первой степени - уравнение вида \( ax + by = c \), где \( a,b,c,x,y \) --- целые числа.

    Пусть \( \textrm{НОД}(a,b) = d \).

    1) Если \( c \vdots d \), то делим на \( d \) правую и левую части уравнения и получаем \( a_1x + b_1y = c_1 \), где \( \textrm{НОД}(a_1, b_1) = 1 \).\\
    2) Если \( c \) не делится на \( d \), то уравнение решений не имеет.\\

    Таким образом, будем рассматривать уравнения (*) \( ax + by = c,\ \textrm{НОД}(a,b) = 1 \).
    
    Так как \( \textrm{НОД}(a, b) = 1 \), то по следствию из алгоритма Евклида \( \exists \) целые \( A,\ B: Aa + Bb = 1 \).\\
    Домножим равенство на \( c: Aca + Bcb = c \).\\
    Видим, что пара целых чисел \( (x_0,y_0) = (Ac, bc) \) является решением уравнения.\\
    Мы нашли частное (одно из) решение нашего уравнения. Найдем все решения \( (x,y) \).\\

    \[ \begin{cases}
        ax_0 + by_0 = c,\\
        ax + by = c.
    \end{cases} \Rightarrow a(x - x_0) + b(y - y_0) = 0,\ a(x - x_0) = -b(y - y_0) \]
    \( \textrm{НОД}(a,b) = 1 \), значит \( (x - x_0) \vdots b \), т.е. \( x - x_0 = bt \) или \( x = x_0 + bt \), где \( t \) --- целое.\\
    Тогда \( y - y_0 = \frac{-a(x - x_0)}{b} = -at \) или \( y = y_0 - at \).\\
    Таким образом, все пары вида \( (x_0 + bt, y_0 - at) \), где \( t \) --- целое, являются решениями (*).

    \textbf{Замечание.} Общее решение диофантова уравнения представляет собой сумму частного решения уравнения и решения соответствующего однородного уравнения (уравнения \( ax + by = 0 \)).

    Легко понять, что решениями однородного уравнения являются все пары вида \( (bt, -at) \), где \( t \) --- целое.

    \textbf{Пример.} 7x - 23y = 131
    Проверка решения: \( c\ \vdots \ \textrm{НОД}(a,b) \Rightarrow \) имеет решения.\\ 
    Можно угадать частное решение (22,1), так как 154 - 23 = 131.\\
    Тогда все решения --- \( (22-33t,1-7t) \), \( t \in \mathbb{Z} \).

    \addtocontents{toc}{\protect\contentsline{section}{\protect\numberline{}Теория сравнений}{}{}}
    \subsection{Сравнения}

    Основная идея теории сравнений заключается в том, что два числа \( a \) и \( b (\in \mathbb{Z}) \), имеющие при делении на \( m \in \mathbb{N} \) один и тот же остаток, обнаруживают целый ряд одинаковых свойств по отношению к \( m \).

    Так по отношению к 2 мы выделяем четные и нечетные числа. Знаем, например, что сумма/разность четных - четное число, произведение четных - четное и т.д.

    \textbf{Определение.} Целые числа \( a \) и \( b \) называются сравнимыми по модулю \( m (a \equiv b (mod\ m)) \), если при делении на \( m \) они дают одинаковые остатки. \textbf{(1)}
    
    \textbf{Пример.} \( 8 \equiv 3 (mod\ 5) \equiv 103 (mod\ 5) \equiv -2 (mod\ 5) \equiv -17 (mod\ 5) \) и т.д.

    \textbf{Определение.} \( a \equiv b (mod\ m) \Leftrightarrow (a - b) \vdots m \). \textbf{(2)}

    Докажем эквивалентность определений 1 и 2.\\
    \( \uparrow \)\\
    1) \textbf{(1)} \( \Rightarrow \) \textbf{(2)}. Пусть остатки одинаковы, т.е. \( a = q_1m + r,\ b = q_2m + r \Rightarrow a - b = m(q_1 - q_2),\ (q_1 - q_2) \in \mathbb{Z} \), т.е. \( (a - b) \vdots m \);\\
    2) \textbf{(2)} \( \Rightarrow \) \textbf{(1)}. От противного.\\
    Пусть остатки разные, т.е. \( a = q_1m + r_1,\ b = q_2m + r_2 \), где \( 0 \leq r_1 < \abs{m},\ 0 \leq r_2 < \abs{m} (-\abs{m} < -r_2 \leq 0) \).\\
    Тогда \( a - b = m(q_1 - q_2) + r_1 - r_2 \) и \( -\abs{m} < r_1 - r_2 < \abs{m} (\abs{r_1 - r_2} < \abs{m} \mathbf{(3)}) \Rightarrow (r_1 - r_2) \vdots m \)\\
    Но тогда по свойству делимости 4, если \( r_1 - r_2 \neq 0 \), то \( \abs{r_1 - r_2} \geq \abs{m} \), противоречие с \textbf{(3)}. Таким образом, \(r_1 = r_2. \downarrow \)

    \subsection{Свойства сравнений}
    1) \( a \equiv a (mod\ m) \)\\
    2) \( a \equiv b (mod\ m) \Rightarrow b \equiv a (mod\ m) \)\\
    3) \( a \equiv b (mod\ m),\ b \equiv c (mod\ m) \Rightarrow a \equiv c (mod\ m) \)
    \[ \uparrow
        \begin{cases}
            (a - b) \vdots m,\\
            (b - c) \vdots m.
        \end{cases} \Rightarrow a - c = \underset{\vdots m}{(a - b)} + \underset{\vdots m}{(b - c)} \vdots m \downarrow \]
    \centerline{Далее считаем, что \( a \equiv b (mod\ m),\ c \equiv d (mod\ m) \)}
    4/5) \( a \pm c \equiv b \pm d (mod\ m) \)
    \[ \uparrow
        \begin{cases}
            (a - b) \vdots m,\\
            (c - d) \vdots m.
        \end{cases}
    \Rightarrow (a + c) - (b + d) = \underset{\vdots m}{(a - b)} + \underset{\vdots m}{(c - d)} \vdots m \downarrow \]
    6) \( ac \equiv bd (mod\ m) \) 
    \[ 
        \begin{cases}
            (a - b) \vdots m,\\
            (c - d) \vdots m.
        \end{cases}
    \Rightarrow ac - bd = ac - bc + bc - bd = c\underset{\vdots m}{(a - b)} + b\underset{\vdots m}{(c - d)} \vdots m \downarrow \]
    7) \( a^k \equiv b^k \)\\
    \textbf{Следствие.} Пусть \(P(x)\) --- любой многочлен с целыми коэффициентами, т.е. \(P(x) = a_nx^n + a_{n - 1}x^{n - 1} + ... + a_0\), тогда из \(x \equiv y (mod\ m) \Rightarrow P(x) \equiv P(y) (mod\ m)\).

    8) Если \(ac \equiv bc (mod\ m)\) и \(\textrm{НОД}(c,m) = 1\), то \(a \equiv b (mod\ m)\).\\
    \(\uparrow ac - bc = c(a - b)\). Так как левая часть делится на \(m\) и \(\textrm{НОД}(c,m) = 1\), то \((a - b) \vdots m \downarrow \)\\
    9) Если \(a \equiv b (mod\ m)\) и \(\exists\ k \in \mathbb{Z}: a = ka_1,\ b = kb_1,\ m = km_1\), то \(a_1 \equiv b_1 (mod\ m_1)\).\\
    \( \uparrow a - c = k(a_1 - b_1) \), т.е. \( k(a_1 - b_1) \vdots km_1 \Rightarrow (a_1 - b_1) \vdots m_1 \downarrow \)

    \textbf{Примеры.}

    1) Признак делимости на 3

    \( \forall n \in \mathbb{N}\quad n = a_k10^k + a_{k - 1}10^{k - 1} + ... + a_1 10 + a_0 \). Так как \(10 \equiv 1 (mod\ 3)\), то \(10^k \equiv 1 (mod\ 3) \Rightarrow n(mod\ 3) = (a_k + a_{k - 1} + ... + a_1 + a_0)(mod\ 3)\).

    2) Признак делимости на 11

    Так как \( 10 \equiv -1 (mod\ 11) \), то \( 10^k \equiv (-1)^k(mod\ 11) \).\\
    Тогда \( n(mod\ 11) = ((-1)^ka_k + ... + a_2 - a_1 + a_0)(mod\ 11) \)

    3) Найти остаток от деления на 3 числа \( n = (1^2 + 1)(2^2 + 1)(3^2 + 1)...(1000^2 + 1) \)

    \( n(mod\ 3) = \{(4^2 + 1) = (1^2 + 1)(mod\ 3),\ (4^2 + 1) = (1^2 + 1)(mod\ 3),\ 1000:3 = 333*3 + 1 \} = (1^2 + 1)^{334}(2^2 + 1)^{333}(3^2 + 1)^{333}(mod\ 3) \equiv (2)^{334}(2)^{333}(1)^{333}(mod\ 3) \equiv (2)^{667}(mod\ 3) \equiv (-1)^{667}(mod\ 3) \equiv -1(mod\ 3) \equiv 2(mod\ 3) \).

    4) При каких натуральных \(n\) число \(8n + 3\) делится на 13?

    То есть при каких \(n\) \(8n + 3 \equiv 0(mod\ 13)\)?

    \( 8n \equiv -3(mod\ 13) \)\\
    \( 8n \equiv 10(mod\ 13) \)\\
    \( 4n \equiv 5(mod\ 13) \)\\
    \( 12n \equiv 15(mod\ 13) \)\\
    \( -n \equiv 2(mod\ 13) \)\\
    \( n \equiv -2(mod\ 13) \)\\
    \( n = 13t - 2,\ t \in \mathbb{N} \) или \( n = 13t + 11,\ t \in \mathbb{N} \)
    
    5) Найти все пары целых чисел \(x\) и \(y\), удовлетворяющих уравнению \(7x - 23y = 131\).

    Избавимся от одного неизвестного: рассмотрим уравнение, например, по модулю 7.

    \( -23y \equiv 131(mod\ 7) \)\\
    \( -2y \equiv 5(mod\ 7) \)\\
    \( 2y \equiv -5(mod\ 7) \)\\
    \( 2y \equiv 2(mod\ 7) \)\\
    \( y \equiv 1(mod\ 7) \Rightarrow y = 7t + 1,\ t \in \mathbb{Z} \)\\
    \( x = \frac{131 + 23y}{7} = \frac{131 + 23*7t + 23}{7} = \frac{154 + 23*7t}{7} = 22 + 23t \)\\
    Ответ: \( (22 + 23t, 1 + 7t), t \in \mathbb{Z} \).

    \subsection{Классификация чисел по данному модулю}
    
    Все числа сравнимые с данным \(a\) (а значит, сравнимые между собой) по модулю \(m\) в один класс.

    Остатками при делении на \(m\) могут быть \(0,1,2,...,m - 1\).

    Значит, можно выделить ровно \(m\) классов по модулю \(m\).

    Класс характеризуется остатком: \( a = mt + r,\ t \in \mathbb{Z},\ 0 \leq r \leq m - 1 \). Фактически, каждый класс --- арифметическая прогрессия со множителем \(m\).

    Выберем произвольным образом по одному числу в каждом классе. Такую группу назовем \textit{полной системой вычетов по модулю \(m\)}(ПСВ(\(m\)). Для данного \(m\) таких систем существует бесконечно много.

    \textbf{Пример.} По \(mod\ 3\): ПСВ(3) = (0,1,2); ПСВ(3) = (10,11,12); ПСВ(3) = (-4,6,-5).
    

\end{document}
