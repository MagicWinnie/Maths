\documentclass{article}

\usepackage[T2A]{fontenc}
\usepackage[utf8]{inputenc}
\usepackage[russian]{babel}
\usepackage{commath}
\usepackage{amsmath}
\usepackage{parskip}
\usepackage{color}
\usepackage{hyperref}
\usepackage[a4paper, left=2.5cm, right=1.5cm, top=2.5cm, bottom=2.5cm]{geometry}

\hypersetup{
    colorlinks=true, %set true if you want colored links
    linktoc=all,     %set to all if you want both sections and subsections linked
    linkcolor=blue,  %choose some color if you want links to stand out
}

\pagenumbering{arabic}

\begin{document}
  % \tableofcontents
  % \thispagestyle{empty}
  % \setcounter{tocdepth}{5}
  % \newpage

  \addtocontents{toc}{\protect\contentsline{section}{\protect\numberline{}Элементы теории чисел. Теория сравнений.}{}{}}
  \title{Элементы теории чисел. Теория сравнений.}
  \author{Ученик 10-4 класса Оконешников Д.Д. по лекции Протопоповой Т.В.}
  \date{14 апреля 2021 г. по лекции от 12 января 2021 г.}
  \maketitle

  \section{Лекция №12}
  
  \textbf{Определение.} \( a \in Z \) и \( b \in Z \backslash \{0\} \) определена операция деления с остатком: разделить целое \( a \) на целое \(b\ (\neq 0) \) с остатком, 
  означает найти такие целые \( q,\ r \in Z \), что \(a=b*q+r,\ 0 \leq r < \abs{b} \).

  \textbf{Определение.} Если при делении с остатком \( r = 0 \), то число \( a \) делится на \( b\ (a \vdots b) \). Число \( b \) при этом называется делителем числа \( a \).

  \textbf{Пример.} \( -7\ \textrm{на}\ 5 \qquad -7 = 5*(-2)+3 \)

  \subsection{Свойства делимости (нацело). ОТА}
  \begin{tabular}{ll}
    \textbf{(1)} Если \( a \vdots c \) и \(b \vdots c\), то \( (a \pm b) \vdots c \) & \textbf{(5)} \( a \vdots b \) и \( b \vdots a \Rightarrow \abs{a} = \abs{b} \)\\
    \( \uparrow
      \begin{aligned}
        &a = cq_1\\
        &b = cq_2
      \end{aligned}
    \Rightarrow (a \pm b) = c(q_1 \pm q_2) \Rightarrow (a \pm b) \vdots c \downarrow \) & \textbf{(6)} \( \forall\ a \in Z \backslash \{0\} \Rightarrow 0 \vdots a \)\\
    \textbf{(2)} \( a \vdots b \Rightarrow ak \vdots b\ (k \in Z)\) & \textbf{(7)} \( \forall\ a \in Z \Rightarrow a \vdots 1 \)\\
    \textbf{(3)} \( a \vdots b,\ b \vdots c \Rightarrow a \vdots c \) & \textbf{(8)} Если \( ab \vdots m \) и \( \textrm{НОД}(a,\ m) = 1, \textrm{ то } b \vdots m \)\\
    \( \uparrow a = bq_1,\ b=cq_2 \Rightarrow a = c*(q_1*q_2) \Rightarrow a \vdots c \downarrow \) & \textbf{(9)} Если \(a \vdots m,\ a \vdots k \textrm{ и НОД}(m,\ k) = 1, \textrm{ то } a \vdots mk \)\\
    \textbf{(4)} Если \( a \neq 0,\ a \vdots b \Rightarrow \abs{a} \geq \abs{b} \) &\\
    \( \uparrow a \vdots b \Leftrightarrow a = b*q \Rightarrow \abs{a} = \abs{b} * \abs{q} \Rightarrow \) от противного, &\\
    если \( \abs{a} < \abs{b} \), то \( \abs{q} = \frac{\abs{a}}{\abs{b}} < \frac{\abs{b}}{\abs{b}} = 1 \Rightarrow \) единственная &\\
    возможность при целом \( q = 0 \), но тогда и \( a = 0 \).&\\
    Противоречие. \( \downarrow \) &\\
  \end{tabular}

  \textbf{Определение.} Натуральное число \( p > 1 \) называется простым, если имеет ровно два натуральных делителя \( (p \textrm{ и } 1) \).

  
  Все остальные натуральные числа называются составными (кроме 1). Единица не является ни простым, ни составным.
  
  \subsubsection{Основная теорема арифметики}
  \textbf{Th.1} (Основная теорема арифметики) Всякое натуральное число \( n > 1 \) может быть представлено в виде \( n = p_1*p_2* ... *p_i \), где \( p_i \) --- простые числа. Это представление едитнственно с точностью до порядка множителей (т.е. если \( n = p_1*p_2*...*p_r = q_1*q_2*...*q_s \), то \( r = s \) и \(q_1,\ q_2,...,\ q_s \) можно перестановкой получить из чисел \( p_1,\ p_2,...,\ p_r \))

  \textbf{(1)} \underline{Докажем существование} 

  Пусть \(n \in N,\ n > 1 \). Среди делителей \( n \) есть числа превосходящие 1 (например, само \( n \)). Пусть \( p_1 \) --- наименьший из таких делителей.

  \( p_1 \) --- простое число (если оно само имело бы делитель \( 1 < a < p_1 \), то \( a \) было бы меньше \( p_1 \) и было бы делителем \( n \) (св-ва 4,3), противоречит тому, что выбран наименьший делитель).

  Итак, \( n = p_1n_1, \textrm{ где } p_1 \textrm{ --- простое, } n_1 \in N \textrm{ и } n_1 < n \) (св-во 4).

  Если \( n_1 > 1 \), то поступим с ним так же, как и числом \( n \), представим его в виде \( n_1 = p_2n_2 \), \( p_2 \) --- простое, \( n_2 \in N, n_2 < n_1 \Rightarrow n = p_1*p_2*n_2 \) и т.д.

  В конце концов, так как \( n_i \in N, i=1,2,3,... \) убывают, то \( \exists\ n_r = 1 \) и процесс обрывается: \( n=p_1*p_2*...p_r \)

  \textbf{(2)} \underline{Докажем существование (единственность)}
  От противного. Если \( \exists \) хоть одно натуральное число, допускающее два существенно различных разложения, то непременно \( \exists \) и \underline{наименьшее} число с таким свойством: \[ m = p_1*p_2*...*p_r = q_1*q_2*...*q_s \qquad (1) \]
  Можем допустить, что \( p_1 \leq p_2 \leq ... \leq p_r; q_1 \leq q_2 \leq ... \leq q_s \).

  А) Заметим, что \( p_1 \neq q_1 \).\\
  Если равны, то разделив (1) на \( p_1 = q_1 \), получили бы два существенно различных разложения на простые множители для числа \( < m \) (Противоречие с тем, что \( m \) --- наименьшее).\\
  На самом деле показали больше: что среди \( q_j \) нет чисел равных какому-либо \( p_i \)

  Б) Из А) \( p_1 < q_1 \) или \( p_1 > q_1 \). Пусть \( p_1 < q_1 \) (для \( p_1 > q_1 \) доказательство строится аналогично).\\
  Рассмотрим целое число: \[ m' = m - p_1*q_2*...*q_s\qquad (2) \]
  Подставляя вместо \( m \) два его разложения, получим: \[ m' = p_1*p_2*...*p_r - p_1*q_2*...*q_s = p_1(p_2*...*p_r - q_2*...*q_s)\qquad (3) \] \[ m' = q_1*q_2*...*q_s - p_1*q_2*...*q_s = (q_1 - p_1)q_2*...*q_s\qquad (4) \]
  Из равенства (4) очевидно \( m' > 0 \). Из равенства (2) \( m' < m \), а значит, для \( m' \) разложение на простые множители --- единственно (с точностью до порядка сомножителей).

  Из (3) \( \Rightarrow p_1 \) входит множителем в \( m' \), значит, из (4) \( p_1 \) входит множителем либо в \( q_1 - p_1 \), либо в \( q_2*...*q_s \). Но последнее невозможно, так как все \( q_j > p_1\ (p_1 < q_1) \) и они простые.
  
  Значит, \( p_1 \) входит множителем в \( q_1 - p_1 \), т.е. \( \mathbf{(q_1 - p_1) \vdots p_1} \) \( \Rightarrow q_1 - p_1 = p_1h \Rightarrow q_1 = p_1(h + 1) \), т.е. \( \mathbf{q_1 \vdots p_1} \), чего быть не может. Противоречие. Ч.Т.Д.

  \subsubsection{Теорема Евклида}
  \textbf{Th.2} (Теорема Евклида) Множество простых чисел бесконечно.

  \( \uparrow \) Доказательство проведем от противного. Предположим, что множество простых чисел конечно, т.е. \( P=\{p_1,p_2,...p_k\} \) --- конечная совокупность простых чисел.

  Рассмотрим число \( p = p_1*p_2*...*p_k + 1 \).

  Заметим, что \( \forall\ i, i=1,2,...,k \) это \( p > p_i \), т.е. \( p \notin P \), значит, оно составное и по ОТА может быть представлено в виде произведения простых множителей.

  Но \( p \) не делится ни на какой \( p_i \) (при делении дает в остатке 1).

  Значит, наше предположение о конечности системы простых чисел неверно. \( \downarrow \)

  \textbf{Утверждение.} Существуют сколь угодно длинные участки натурального ряда, вовсе не содержащие простых чисел

  \( \uparrow \) Действительно, пусть \( n \in N, n > 1 \). Рассмотрим ряд чисел: \( n! + 2,n! + 3,...,n! + n \).
  
  \begin{tabular}{ll}
    \( n = 2: 2! + 2 \) --- одно число в ряду; &\\
    \( n = 3: 3! + 2, 3! + 3 \) --- два числа в ряду; & чем больше \( n \), тем больше в ряду\\
    \( n = 4: 4! + 2, 4! + 3, 4! + 4 \) --- три числа в ряду; & чисел (\( n - 1 \) число).\\
    и т.д. &\\
  \end{tabular}

  В этом ряду нет ни одного простого числа, так как \( n! + 2 \) делится на 2, \( n! + 3 \) делится на 3, \( n! + n \) делится на n. Таким образом, при больших \( n \) такие участки натурального ряда могут быть очень большими. \( \downarrow \) 

  \subsubsection{Теорема Эйлера}
  \textbf{Th.3} (Теорема Эйлера) Пусть \( \tau(n) \) --- количество простых чисел \( \leq n \). Тогда \[ \frac{\tau(n)}{n} \xrightarrow[n \rightarrow \infty]{} 0 \]
  Понятно, что \( \tau(n) \) увеличивается (т.е. \( \rightarrow \infty \)) при \( n \rightarrow \infty \) (это означает, что простые числа встречаются все реже и реже).

  Мы показали, что любое натуральное число мы можем представить в виде произведения простых множителей (и такое представление единственно с точностью до перестановки множителей): \( n = p_1*p_2*...*p_r,\ p_1 \leq p_2 \leq ... \leq p_r \). Используя обозначение степени, можем записать так: \[ \mathbf{n = p_1^{a_1}*p_2^{a_2}*...*p_k^{a_k}}, \textrm{ (каноническое разложение) } \] 
  \[ \textrm{где } p_1 < p_2 < ... < p_k \textrm{ --- простые, } a_1, a_2, ..., a_k \textrm{ --- натуральные числа.} \]
  \textit{Замечание.} Бывает полезно записать в разложение \underline{все} простые числа \( \leq p_k \) и использовать показатель равный 0.

  Если число \( m \) является делителем \( n \), то несложно понять, что \( \mathbf{m = p_1^{\beta_1}*p_2^{\beta_2}*...*p_k^{\beta_k}} \), где \( 0 \leq \beta_i \leq a_i \).

  Можно посчитать число всех натуральных делителей числа \( n \). Любой делитель \( n \) имеет следующую структуру: \( \mathbf{m = p_1^{0,1,2,...,a_1}*p_2^{0,1,...a_2}*...*p_k^{0,1,...,a_k}} \)

  Для первого множителя \( (a_1 + 1) \) возможность для второго \( (a_2 + 1) \) возможностей и т.д. Таким образом, число всех делителей \( (a_1 + 1)*(a_2 + 1)*...*(a_k + 1) \).

  \textbf{Пример.} Сколько делителей у числа 120 (включая 1 и само число)?

  \begin{tabular}{r|l}
    120 & 2\\
    60 & 2\\
    30 & 2\\
    15 & 3\\
    5 & 5\\
    1 & \\
  \end{tabular}

  \( 120 = 2^3 * 3^1 * 5^1 \). Значит, число всех делителей \( = (3 + 1)*(1 + 1)*(1 + 1) = 4*2*2 = 16 \).

  \textbf{Определение.} \( d \) --- общий делитель \( a \) и \( b \Leftrightarrow \mathbf{a \vdots d } \) и \( \mathbf{b \vdots d} \).

  \textbf{Определение.} Наибольший общий делитель чисел \( a \) и \( b \) обозначается \( \textrm{НОД}(a, b) \).

  \textbf{Определение.} Наименьшее общее кратное \( \textrm{НОК}(a, b) = k \) --- наименьшее натуральное число такое, что \( \mathbf{k \vdots a} \) и \( \mathbf{k \vdots b} \).

  Пусть \( \mathbf{a = p_1^{a_1} * p_2^{a_2} * ... * p_k^{a_k}},\ \mathbf{b = p_1^{\beta_1} * p_2^{\beta_2} * ... * p_k^{\beta_k}} \)

  Здесь использовали показатель 0 для тех простых множителей, которые входят только в одно из разложений.

  Тогда \[ \textrm{НОД}(a, b) = p_1^{min(a_1, \beta_1)} * p_2^{min(a_2, \beta_2)} * ... * p_k^{min(a_k, \beta_k)} \]
  \[ \textrm{НОК}(a, b) = p_1^{max(a_1, \beta_1)} * p_2^{max(a_2, \beta_2)} * ... * p_k^{max(a_k, \beta_k)} \]
  \[ \textrm{НОД}(a, b)*\textrm{НОК}(a, b) = a * b \]

  \textbf{Пример.} \( a = 2 * 3^3 * 5^2 * 7,\ b = 2^2 * 3 * 7^2 * 11 \Rightarrow a = 2^1 * 3^3 * 5^2 * 7^1 * 11^0,\ b = 2^2 * 3^1 * 5^0 * 7^2 * 11^1 \Rightarrow \textrm{НОД}(a, b) = 2^1 * 3^1 * 5^0 * 7^1 * 11^0,\ \textrm{НОК}(a, b) = 2^2 * 3^3 * 5^2 * 7^2 * 11^1 \).

  Чтобы получить каноническое разложение полезно помнить признаки делимости.
  
  1) на 2 и 5. Легко.\\
  2) на 4. \( n = \overline{a_ka_{k-1}...a_1a_0} = 100 * \overline{a_ka_{k-1}...a_2} + \overline{a_1a_0}.\quad 100 \vdots 4.\ \Rightarrow n \vdots 4 \Leftrightarrow \overline{a_1a_0} \vdots 4 \).\\
  3) на 8. \( n \vdots 8 \Leftrightarrow \overline{a_2a_1a_0} \vdots 8 \).\\
  4) на 3. \( n = \overline{a_ka_{k-1}...a_1a_0} = a_k10^k + a_{k-1}10^{k-1} + ... + a_1 10 + a_0 = a_k(\underbrace{999...9}_k + 1) + a_{k-1}(\underbrace{999...9}_{k-1} + 1) + ... + a_1(9 + 1) + a_0 = (a_k\underbrace{999...9}_k + a_{k-1}\underbrace{999...9}_{k-1} + ... + a_1 9) + (a_k + a_{k-1} + ... + a_1 + a_0) \)\\
  Аналогично для 9.\\
  5) на 6. \( n \vdots 2 \) и \( n \vdots 3 \Rightarrow \) (так как 2 и 3 взаимно просты) \( n \vdots 6 \)\\
  6) на 11. \( n = \overline{a_ka_{k-1}...a_1a_0} = \)
\end{document}
