\documentclass{article}

\usepackage[T2A]{fontenc}
\usepackage[utf8]{inputenc}
\usepackage[russian]{babel}
\usepackage{commath}
\usepackage{amsmath}
\usepackage{amsfonts}
\usepackage{mathtools}
\usepackage{amssymb} 
\usepackage{parskip}
\usepackage{titling}
\usepackage{color}
\usepackage{hyperref}
\usepackage[a4paper, left=2.5cm, right=1.5cm, top=2.5cm, bottom=2.5cm]{geometry}

\setlength{\droptitle}{-3cm}
\hypersetup{
    colorlinks=true, %set true if you want colored links
    linktoc=all,     %set to all if you want both sections and subsections linked
    linkcolor=blue,  %choose some color if you want links to stand out
}

\pagenumbering{arabic}

\begin{document}
    \title{Числовые последовательности и их пределы}
    \author{Ученики 10-4 класса Оконешников Д.Д. и Паньков М.А. \\по лекции к.ф.-м.н. Протопоповой Т.В.}
    \date{от 05 мая 2021 г.}
    \maketitle

    \section{Лекция №26}

    \textbf{Теорема Больцано-Вейерштрасса} 

    Если \( \{x_n\} \) \underline{ограничена}, то \( \exists \)-ет хотя бы одна сходящаяся последовательность.

    \(\uparrow\)
    \begin{enumerate}
        \item Раз ограничена; то \(\exists\ M\ :\ \forall n\ \abs{x_n} \leq M\), т.е. \(I_0 = [-M; M]\), то \(\forall n\ x_n \in I_0\)\\
        \( d_0 = 2M \)

        \item \( \frac{I_0}{2} \) (делим \( I_0 \) пополам)\\
        Пусть \( I_1 \)-та половина, в которой содержится \( \infty \) число членов последовательности \(x_n\) (если в обеих, то выбираем любую из них) \( I_1 = \frac{I_0}{2} \)\\
        \( d_1 = \frac{2M}{2} \)

        \item \( \frac{I_1}{2} \Rightarrow\), та часть, в которой содержится \( \infty \) число членов последовательности \( \{x_n\} \) (если в обеих, то любую из) в \( I_2\ b_2 = x_{n2} \in I_2\ n_2>n_1 \)\\
        \( d_1 = \frac{2M}{2^2} \)
        %IMAGE TODO
        \item и т.д. \(I_0 \supset I_1 \supset I_2 \supset ... \supset I_k \supset .....\)
        \\k)\(d_k = \frac{2M}{2^k} \longrightarrow_{k \rightarrow \infty} 0\)
    \end{enumerate}
    по Т. о вложенных промежутках \(\exists !\ B\ :\ B \in I_k\ \forall k\)
    \\Покажем, что \(\{b_k\} \xrightarrow[k \rightarrow \infty]{} B\)
    
    \( \forall \varepsilon > 0 \exists\ K: \forall k > K:\) 

    \(\abs{b_k - B} < \varepsilon\)

    \( \abs{b_k - B} < \abs{I_k} = \frac{2M}{2^k} < \frac{2M}{k} < \varepsilon \)

    \( 2^k > k\ K = [\frac{2M}{\varepsilon}] + 1\)

    \(\downarrow\)
    
    \subsection{Верхний и нижний предел последовательности}
    \subsubsection{Верхний предел}
    \textbf{Определение.} Число M будем называть верхним пределом последовательности \(x_n\), если
    \begin{enumerate}
        \item \( \exists\ x_{nk}: x_{nk}  \xrightarrow[k \rightarrow \infty]{} M\)
    
        \item \(\forall x'_{nk} \rightarrow M'\ M' \leq M\)

    \end{enumerate}
    \( \uparrow \)

    \textbf{Обозначение.} \(\quad \overline{\lim_{n \rightarrow \infty}}\ x_n = \overline{\lim} x_n = M \)

    \textbf{Пример.}
    \begin{enumerate}
        \item \(x_n = (-1)^n \Rightarrow \lim{x_n} = 1\)
        
        \item \(x_n = \sin{\frac{\pi n}{2}} \Rightarrow \lim{x_n} = 1\)
    
        \item \(x_n = (n)^{-1} \Rightarrow \overline{\lim}x_n = +\infty\)\\
        \( 1; 2; \frac{1}{3}; 4; \frac{1}{5}; 6; \frac{1}{7}; ... \)
    \end{enumerate}
    
    \textbf{Замечание.} 
    \begin{enumerate}
        \item Если \(x_n\) не ограничена сверху, то \(\overline{\lim}x_n = +\infty\)
        
        \item Если \(\overline{\lim}x_n = a\)(a --- конечное число)
        %IMAGE HERE
    \end{enumerate}
    % ---------------------------
    \subsubsection{Нижний предел}
    \textbf{Определение.} Число m будем называть нижним пределом последовательности \(x_n\), если
    \begin{enumerate}
        \item \( \exists\ x_{nk}: x_{nk}  \xrightarrow[k \rightarrow \infty]{} m\)
    
        \item \(\forall x'_{nk} \rightarrow m'\ m' \leq m\)

    \end{enumerate}

    \textbf{Обозначение.} \(\quad \underline{\lim_{n \rightarrow \infty}}\ x_n = \underline{\lim} x_n = M \)

    \textbf{Пример.}
    \begin{enumerate}
        \item \(x_n = (-1)^n \Rightarrow \lim{x_n} = 1; \underline{\lim}x_n=-1\)
        
        \item \(x_n = \sin{\frac{\pi n}{2}} \Rightarrow \lim{x_n} = 1 \underline{\lim}x_n=-1\)
    
        \item \(x_n = (n)^{-1} \Rightarrow \overline{\lim}x_n = +\infty;\underline{\lim}x_n=0\)\\
        \( 1; 2; \frac{1}{3}; 4; \frac{1}{5}; 6; \frac{1}{7}; ... \)
    \end{enumerate}
    
    \textbf{Замечание.} 
    \begin{enumerate}
        \item Если \(x_n\) не ограничена снизу, то \(\underline{\lim}x_n = -\infty\)
        
        \item Если \(\underline{\lim}x_n = a\)(a --- конечное число)
        %IMAGE HERE
    \end{enumerate}
    
    \textbf{Теорема.} У всякой \(\{x_n\}\ \exists\ \overline{\lim}\) и \(\underline{\lim}\).

    \(\uparrow\)
    \begin{enumerate}
        \item \(x_n\) --- не ограничена сверху, то \(\overline{\lim}x_n = +\infty\)
        \item \(x_n\) --- не ограничена снизу, то \(\underline{\lim}x_n = -\infty\)
        \item \(x_n\) ограничена сверху и снизу
    \end{enumerate}

    дальше как в Т. Больцано-Вейерштрасса, только 3.1) если ищем \( \overline{\lim} \), то всегда беру правую половинку, если возникает выбор.
    \\3.2) Для \(\underline{\lim}\), то беру левую половинку

    \( \downarrow \)

    \textbf{Замечание.} Очевидно, что \( \underline{\lim}x_n \leq \overline{\lim}x_n \)

    \textbf{Теорема.} \( \exists\ \lim_{n \rightarrow \infty} x_n = A \) (конечн, \(+\infty\), \(-\infty\)) \( \Leftrightarrow \overline{\lim}x_n = \underline{\lim}x_n = A \)

    \(\uparrow\)
    "\(\Rightarrow\)" пусть \(\lim_{n \rightarrow \infty}{x_n} = A\)

    \begin{enumerate}
        % IMAGE HERE
        \item \( A = +\infty \), т.е. \( lim_{n \rightarrow \infty}x_n = +\infty \)\\
        \( \Rightarrow \underline{lim}x_n = +\infty \Rightarrow \overline{lim}x_n = +\infty \)
        % IMAGE HERE        
        \item \(A = -\infty\); т.е. \(\lim_{n \rightarrow \infty}{x_n} = -\infty \Rightarrow \overline{\lim}x_n = -\infty \Rightarrow \underline{\lim}x_n = -\infty\)
        % IMAGE HERE
        \item Если \( \lim_{n \rightarrow \infty}x_n = A \) (A --- конечное число) \( \Rightarrow \overline{\lim}x_n = \underline{\lim}x_n = A \) уже доказывали
    \end{enumerate}

    "\( \Leftarrow \)"
    \begin{enumerate}
        \item \(\lim x_n = +\infty\)
        % IMAGE HERE
        т.е. \(\lim{n \rightarrow \infty}{x_n} = +\infty\)
        
        \item \( \overline{\lim}x_n = -\infty \) 
        % IMAGE HERE
        \( \Rightarrow \lim x_n = -\infty \)

        \item A --- конечный \(\overline{\lim} x_n = \underline{\lim} x_n = A\)
        %IMAGE HERE
    \end{enumerate}
    Надо \( \forall \varepsilon > 0\ \exists N\ \forall n > N\ A - \varepsilon <_{\textrm{из } \underline{\lim}x_n = A} x_n <_{\textrm{из } \overline{\lim}x_n = A} A + \varepsilon \), т.е. \( \lim_{n \rightarrow \infty}x_n = A \)
    
    \( \downarrow \)

    \subsection{Фундаментальные последовательности. Критерий Коши}

    \(\lim_{n \rightarrow \infty}{x_n} = a\)

    \( \forall\ \varepsilon > 0\ \exists N: \forall n > N \)

    \( \abs{x_n - a} < \varepsilon \)

    \( a - \varepsilon < x_n < a + \varepsilon \)
    % IMAGE HERE

    \( (a - \varepsilon; a + \varepsilon) \)

    Т.е. \( x_n \) --- такие, что для любого \( \varepsilon > 0 \) я могу их всех(n > N) "закрыть" интервалом длины \(2\varepsilon\)

    \textbf{Определение.} Последовательность \( x_n \) называется фундаментальной(удовлетворяет условию Коши), если \( \forall \varepsilon > 0\ \exists N: \forall n, m > N,\ \abs{x_n - x_m} < \varepsilon \)
    
    \(x_m - \varepsilon < x_n < x_m + \varepsilon\)

    \textbf{Теорема. Критерий Коши.} Числовая последовательность \( \{x_n\} \) сходится \( \Leftrightarrow \forall \varepsilon > 0\ \exists N(\varepsilon): \forall n, m > N: \abs{x_n - x_m} < \varepsilon \) 

    \(\uparrow\)
    
    "\(\Rightarrow\)" пусть сходится (\(\lim_{n \rightarrow \infty}{x_n} = a\) --- конечное число) \(\forall \varepsilon > 0\ \exists N : \forall n > N: \abs{x_n - a} < \frac{\varepsilon}{2} \)

    \( \abs{x_n - x_m} = \abs{x_n - a + a - x_m} \leq \abs{x_n - a} + \abs{a - x_m} = \abs{x_n - a} + \abs{x_m - a} <_{n > N; m > N} \frac{\varepsilon}{2} + \frac{\varepsilon}{2} = \varepsilon \)

    "\(\Leftarrow\)" 1) покажем, что \(\{x_n\}\) --- ограничена \(\forall \varepsilon > 0\ \exists N_0 : \forall n,m > N_0,\ \abs{x_n - x_m} < \varepsilon\)
    
    Пусть \(\varepsilon = 1\)
    \\\(\abs{x_n} - \abs{x_m} \leq \abs{x_n - x_m} < 1\)
    \\\( \Downarrow \)
    \\\( \abs{x_n} < 1 + \abs{x_m} \)

    Фиксированный \( m \Rightarrow \forall n > N_0 \) рассмотрим \( M = \max{\{ 1 + \abs{x_m}; \abs{x_1}; \abs{x_2}; \abs{x_3}; ... ; \abs{x_{N_0}} \}} \)
    
    2) По Т. Больцано-Вейерштрасса из \(\{x_n\}\) можно выделить сходящиеся последовательности
    \\\(\exists x_{nk}\ :\ x_{nk} \xrightarrow[k \rightarrow \infty]{} a \)

    Покажем, что \( x_n \rightarrow a \)
\end{document}