\documentclass{article}

\usepackage[T2A]{fontenc}
\usepackage[utf8]{inputenc}
\usepackage[russian]{babel}
\usepackage{commath}
\usepackage{amsmath}
\usepackage{amsfonts}
\usepackage{mathtools}
\usepackage{amssymb} 
\usepackage{parskip}
\usepackage{titling}
\usepackage{color}
\usepackage{hyperref}
\usepackage[a4paper, left=2.5cm, right=1.5cm, top=2.5cm, bottom=2.5cm]{geometry}

\setlength{\droptitle}{-3cm}
\hypersetup{
    colorlinks=true, %set true if you want colored links
    linktoc=all,     %set to all if you want both sections and subsections linked
    linkcolor=blue,  %choose some color if you want links to stand out
}

\pagenumbering{arabic}

\begin{document}
    \title{Числовые последовательности и их пределы}
    \author{Ученик 10-4 класса Паньков М.А. по лекции к.ф.-м.н. Протопоповой Т.В.}
    \date{от 12 мая 2021 г.}
    \maketitle

    \section{Лекция №27}

    %\textbf{Теорема. Критерий Коши.} Числовая последовательность \( \{x_n\} \) сходится \( \Leftrightarrow \forall \varepsilon > 0\ \exists N(\varepsilon): \forall n, m > N: \abs{x_n - x_m} < \varepsilon \)
    Покажем, что \( x_n \rightarrow a \), т.е. \(\forall \varepsilon > 0\ \exists N_0\ :\ \forall n, m > N_0,\ \abs{x_n - x_m} < \varepsilon \)

    Имеем:\\
    \(\forall \varepsilon > 0\ \exists N_0\ :\ \forall n, m > N_0,\ \abs{x_n - x_m} < \frac{\varepsilon}{2}\) и\\
    \(x_{nk} \rightarrow a\), т.е. \(\forall \varepsilon > 0\ \exists K_0\ :\ \forall k > K_0,\ \abs{x_{nk} - a} < \frac{\varepsilon}{2}\)

    \(\abs{x_n - a} = \abs{x_n - x_{nk} + x_{nk} - a} \leq \abs{x_n - x_{nk}} + \abs{x_{nk} - a} < \frac{\varepsilon}{2} + \frac{\varepsilon}{2} = \varepsilon\)
    \\ \(n_k > N_0\\ k > K_0\ n_k > n_{K_0}\)
    \\\(N = \max(N_0, n_{K_0})\)
    \\\(\downarrow\)
    
    \textbf{Пример.} \(x_n = \frac{\sin{1}}{2} + \frac{\sin{2}}{2^2} + ... + \frac{\sin{n}}{2^n}\) --- доказать сходимость

    \quad \quad \textbf{Замечание.} Другая форма условия Коши: \(\forall \varepsilon > 0\ \exists N_0\ :\ \forall n > N_0;\ \forall p > 0,\ \abs{x_n - x_{n+p}} < \varepsilon\)

    \(\abs{x_n - x_{n+p}} = \abs{\frac{\sin{1}}{2} + \frac{\sin{2}}{2^2} + ... + \frac{\sin{n}}{2^n} - (\frac{\sin{1}}{2} + \frac{\sin{2}}{2^2} + ... + \frac{\sin{n}}{2^n} + ...
    + \frac{\sin{n+p}}{2^{n+p}})} = \abs{\frac{\sin{n}}{2^n} + \frac{\sin{n+1}}{2^{n+1}} + ... + \frac{\sin{n+p}}{2^{n+p}}} \leq \frac{\abs{\sin{n+1}}}{2^{n+1}} + \frac{\abs{\sin{n+2}}}{2^{n+2}} + ... + \frac{\abs{\sin{n+p}}}{2^{n+p}} \leq 
    \frac{1}{2^{n+1}} + \frac{1}{2^{n+2}} + ... + \frac{1}{2^{n+p}} = \frac{\frac{1}{2^{n+p+1}} - \frac{1}{2^{n+1}}}{\frac{1}{2} - 1} = \frac{\frac{1}{2^{n+ \not 1}}(1 - \frac{1}{2^p})}{\not \frac{1}{2}} < \frac{1}{2^n} < \frac{1}{n} < \varepsilon\)
    %Типо подведение итогов; можно сделать ссылку на теоремы
    \begin{enumerate}
        \item Теорема Вейерштрасса
        \\\(\Downarrow\)
        \item Принцип вложенных промежутков
        \\\(\Downarrow\)
        \item Теорема Больцано-Вейерштрасса
        \\\(\Downarrow\)
        \item Критерий Коши
    \end{enumerate}
\end{document}