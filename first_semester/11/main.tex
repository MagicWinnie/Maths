\documentclass{article}
% PKGS START
\usepackage[utf8x]{inputenc}
\usepackage[english,russian]{babel}
\usepackage{cmap}
\usepackage{commath}
\usepackage{amsmath}
\usepackage{amsfonts}
\usepackage{mathtools}
\usepackage{amssymb} 
\usepackage{parskip}
\usepackage{titling}
\usepackage{color}
\usepackage{hyperref}
\usepackage{cancel}
\usepackage{enumerate}
\usepackage{graphicx}
\usepackage[a4paper, left=2.5cm, right=1.5cm, top=2.5cm, bottom=2.5cm]{geometry}
% PKGS END
% INIT START
\graphicspath{ {./images/} }
\setlength{\droptitle}{-3cm}
\hypersetup{
    colorlinks=true, %set true if you want colored links
    linktoc=all,     %set to all if you want both sections and subsections linked
    linkcolor=blue,  %choose some color if you want links to stand out
}

\pagenumbering{arabic}
% INIT END
\begin{document}
    В общем случае:
    \textbf{Определение 1.} Пусть имеются предметы \(n\) видов и из них составлются наборы, содержащие k элементов. 
    Два набора считаются одинаковыми, если имеют одинаковый состав. 
    Такие наборы назовем сочетаниями с повторениями из \(n\) по \(k\) элементов.

    Число сочетаний с повторениями обозначается \(\overline{C_n^k}\)

    \textbf{Задача.} Найдем \(\overline{C_n^k}\)

    \(\uparrow\) 1) \((k_1, k_2, ..., k_n): k_1 + k_2 + ... + k_n = k\)

    2) \((k_1, k_2, ..., k_n) \leftrightarrow (0001111...10)\) последовательность из \(k\) единиц и \(n-1\) нуля.

    3) \(\overline{C_n^k} = P(k, n-1) = \frac{(k+n-1)!}{k!(n-1)!} = C_{k+n-1}^k\) \(\downarrow\)
    
    %11 лекция 9:00 таблица - обобщение материала по комбинаторике

    \section{Бином Ньютона (Формула Ньютона для бинома)}

    Формулы сокращенного умножения для \((a + b)^n\)

    \textbf{Теорема.} \(\forall\ a\) и \(b\) и произвольного \(n \in \mathbb{N}\) справедлива формула \((a + b)^n = \sum\limits_{k=0}^n C_n^k a^{n-k} b^k\) или, подробнее, \((a + b)^n = C_n^0 a^n + C_n^1 a^{n-1} b + C_n^2 a^{n-2} b^2 + ... + C_n^{k} a^{n-k} b^k + ... + C_n^n b^n\).

    \(\uparrow\) 1) \(n = 1\) \(C_1^0 a + C_1^1 b = \frac{1!}{0!1!} a + \frac{1!}{1!0!} b = a + b\)
\end{document}