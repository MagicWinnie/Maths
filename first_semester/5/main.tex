\documentclass{article}
% PKGS START
\usepackage[utf8x]{inputenc}
\usepackage[english,russian]{babel}
\usepackage{cmap}
\usepackage{commath}
\usepackage{amsmath}
\usepackage{amsfonts}
\usepackage{mathtools}
\usepackage{amssymb} 
\usepackage{parskip}
\usepackage{titling}
\usepackage{color}
\usepackage{hyperref}
\usepackage{cancel}
\usepackage{enumerate}
\usepackage{multicol}
\usepackage{graphicx}
\usepackage[a4paper, left=2.5cm, right=1.5cm, top=2.5cm, bottom=2.5cm]{geometry}
% PKGS END
% INIT START
\graphicspath{ {./images/} }
\setlength{\droptitle}{-3cm}
\hypersetup{
    colorlinks=true, %set true if you want colored links
    linktoc=all,     %set to all if you want both sections and subsections linked
    linkcolor=blue,  %choose some color if you want links to stand out
}

\pagenumbering{arabic}
% INIT END
\begin{document}
    \section{Отображения и функции}

    \subsection{Простейшие преобразования графиков функций}

    \textbf{Цель.} Как, зная график функции \(y=f(x)\), построить график функции \(y=Af(ax+b)+B\)?

    \begin{enumerate}
        \item Симметрия относительно оси \(OX\)
        \item Симметрия относительно оси \(OY\)
        \item Центральная симметрия

        \textbf{Определение 1.} Функция \(f: x \rightarrow f(x)\) называется чётной, если
        \begin{enumerate}
            \item \(D_f\) симметрична относительно нуля
            \item \(f(x)=f(-x)\)
        \end{enumerate}

        \textbf{Определение 2.} Функция \(f: x \rightarrow f(x)\) называется нечётной, если
        \begin{enumerate}
            \item \(D_f\) симметрична относительно нуля
            \item \(f(x)=-f(-x)(f(-x)=-f(x))\)
        \end{enumerate}

        \textbf{Теорема 1.} Всякую функцию с симметричной относительно нуля областью определения можно представить в виде суммы чётной и нечётной функции.

        \(\uparrow\) Рассмотрим функции \(g(x)=\frac{f(x)+f(-x)}{2}\) и \(h(x)=\frac{f(x)-f(-x)}{2}\).
        Область определения функций \(g(x)\) и \(h(x)\) совпадает с областью определения функции \(f(x)\).

        Для \(g(x)\): \(g(-x)=\frac{f(-x)+f(x)}{2}=g(x)\), значит, \(g(x)\) --- чётная.

        Для \(h(x)\): \(h(-x)=\frac{f(-x)-f(-(-x))}{2}=-h(x)\), значит, \(h(x)\) --- нечётная.

        Заметим, что \(g(x)+h(x)=\frac{f(x)+f(-x)}{2}+\frac{f(x)-f(-x)}{2}=\frac{2f(x)}{2}=f(x)\), т.е. ч.т.д.\(\downarrow\)

        \textbf{Замечание.} Конечно, вид функций \(g(x), h(x)\) появился не "вдруг". Мы ищем чётную \(g(x)\) и нечётную \(h(x)\) такие, что
        \begin{equation}\label{1}
            f(x)=g(x)+b(x)
        \end{equation}
        Из свойств \(g(x)\) и \(h(x)\) имеем
        \begin{equation}\label{2}
            f(-x)=g(x)-h(x)
        \end{equation}
        Складывая \ref{1} и \ref{2}, находим представление для \(g(x)\), а вычитая \ref{2} из \ref{1}, --- представление для \(h(x)\).
        \item Перенос графика вверх/вниз
        %image at 13:50

        \((x,y)\rightarrow(x, y+B),\\ \Gamma_f=\{(x,f(x))\} \rightarrow \Gamma_g=\{(x,f(x)+B)\}\), т.е.\\
        \(g\): \(x\rightarrow f(x)+B\) и \(g(x)=f(x)+B\)

        \(B>0\) --- перенос вверх на \(B\) единиц.\\
        \(B<0\) --- перенос вниз.\\
        \(B=0\) --- тождественное преобразование.
        \item Перенос графика вправо/влево
        %image at 16:07

        На рисунке \(b>0\), перенос влево \((x,y)\rightarrow (x-b,y),\\
        \Gamma_f=\{(x,f(x))\}\rightarrow \Gamma_g = \{(x-b,f(x))\}=\{(x',f(x'+b))\}\), т.е.\\
        \(g\): \(x\rightarrow f(x+b)\) и \(g(x)=f(x+b)\)

        \(b>0\) --- перенос влево на \(b\) единиц.\\
        \(b<0\) --- перенос вправо.\\
        \(b=0\) --- тождественное преобразование.
        \item Растяжение/сжатие вдоль \(OY\)(от/к \(OX\))
        %image at 52:33

        \((x,y)\rightarrow (x,Ay)\).\\
        Рассмотрим случай \(A>0\)(если \(A<0\), то сначала симметрия относительно \(OX\)):\\
        \(\Gamma_f=\{(x,f(x))\}\rightarrow \Gamma_g=\{(x,Af(x))\}\), т.е.\\
        \(g\): \(x \rightarrow Af(x)\) и \(g(x)=Af(x)\)

        \(A>1\) --- растяжение вдоль \(OY\) в \(A\) раз.\\
        \(0<A<1\) --- сжатие.\\
        \(A=0\) --- тождественное преобразование.
        \item Растяжение/сжатие вдоль \(OX\)(от/к \(OY\))
        %image at 52:33
        
        \((x,y)\rightarrow (ax,y)\).\\
        Рассмотрим случай \(a>0\)(если \(a<0\), то сначала симметрия относительно \(OY\)):\\
        \(\Gamma_f=\{(x,f(x))\}\rightarrow \Gamma_g=\{(ax,f(x))\}=\{(x',f(x'/a))\}\), т.е.\\
        \(g\): \(x \rightarrow f(x/a)\) и \(g(x)=f(x/a)\)

        \(a>1\) --- сжатие вдоль \(OX\) в \(a\) раз.\\
        \(0<a<1\) --- растяжение вдоль \(OX\).\\
        \(a=0\) --- тождественное преобразование.
    \end{enumerate}
    
    \textbf{Способ получения графика функции \(\textit{\textbf{y=Af(ax+b)+B}}\), при данном графике \(\textit{\textbf{y=f(x)}}\)}

    \begin{enumerate}
        \item Из \(f(x)\) сжатием/растяжением вдоль \(OX\) получаем \(y_1=f(ax)\). Здесь же симметрия относительно \(OY\), если \(a<0\).
        \item Сжатием/растяжением вдоль \(OY\) получаем \(y_2=Af(ax)\). Здесь же симметрия относительно оси \(OX\), если \(A<0\).
        \item Параллельный перенос на вектор \((-b/a, B)\).
    \end{enumerate}

    \subsection{Периодические функции}
    %image at 22:44
    
    \textbf{Определение.} Функция \(f(x)\) называется периодической, если \(\exists\) хотя бы одно \(T \not = 0\)(период) такое, что выполнены следующие условия:
    \begin{enumerate}
        \item \(x \in D_f \Leftrightarrow x + T \in D_f\)
        \item \(f(x)=f(x+T)\)
    \end{enumerate}

    \subsubsection{Свойства периодических функций}

    Пусть \(f(x)\) периодическая функция с периодом \(T\).
    \begin{enumerate}
        \item \(\forall k \in \mathbb{Z}, k \not = 0\) число \(kT\) является периодом.
        
        \(\uparrow f(x)=f(x+T)=f((x+T)+T)=f(x+2T)=f((x+2T)+T)=...=f(x+kT) \downarrow\)
        \item Периодическая функция не может иметь "конечную" область определения(О.О. не может являться ограниченным множеством),
        но может состоять из бесконечного числа ограниченных множеств.
        \item Периодическая функция принимает каждое своё значение бесконечное число раз.
    \end{enumerate}

    \subsection{Линейная функция и её график}

    Рассматриваем функцию \(y=kx+b\).

    График этой функции можем получить путём простейших преобразований из графика функции \(y=x\).

    \(b>0\) --- сдвиг прямой \(y=x\) вверх, \(b<0\) --- сдвиг прямой \(y=x\) вниз,

    \(k\) отвечает за возрастание/убывание функции: если \(k>0\), то функция строго возрастает, если \(k<0\), то функция строго убывает, если \(k=0\), то функция постоянна.

    \textbf{Утверждение.} Две точки определяют прямую.

    Известно, что точки \((x_1, y_1)\) и \((x_2, y_2)\) принадлежат прямой. Определить уравнение этой прямой.

    \(\begin{cases}
        kx_1+b=y_1\\
        kx_2+b=y_2
    \end{cases}\)
    Можно выразить \(b\) из первого и подставить во второе.
\end{document}