\documentclass{article}

\usepackage[utf8x]{inputenc}
\usepackage[english,russian]{babel}
\usepackage{cmap}
\usepackage{commath}
\usepackage{amsmath}
\usepackage{amsfonts}
\usepackage{mathtools}
\usepackage{amssymb}
\usepackage{parskip}
\usepackage{titling}
\usepackage{color}
\usepackage{hyperref}
\usepackage{cancel}
\usepackage{enumerate}
\usepackage{multicol}
\usepackage{graphicx}
\usepackage{docmute}
\usepackage[font=small,labelfont=bf]{caption}
\usepackage[a4paper, left=2.5cm, right=1.5cm, top=2.5cm, bottom=2.5cm]{geometry}

\graphicspath{ {./images/} }
\setlength{\droptitle}{-3cm}
\hypersetup{ colorlinks=true, linktoc=all, linkcolor=blue }
\pagenumbering{arabic}

\begin{document}
    \subsection{Объём тел вращения} %пункт 10

    %03:03 graph

    \(y = f(x)\) на \([a;b]\)
    \(f(x) ---\) непрерывна
    \(f(x) \geq 0 \\ f(x) \uparrow\)
    \(\Downarrow\)
    \(\begin{cases}
        y = f(x)\\
        y = 0\\
        x = a\\
        x = b
    \end{cases}\)

    \(\Rightarrow T ---\) тело вращения \( V(t) = \pi \int_{a}^{b}f^2(x)dx\)

    \(1. \exists! V(T)\)
    %08:35 
    %сделать таблицу тут —>
    \(a = x_0 x_1, x_2 ... x_k, x_{k+1} ... x_n = b   \Delta x_n = x_{n+1} - x_k = \frac{b- a}{n}\)
    плоскость \( \pi_0, \ \pi_1, \ \pi_2 ... \pi_k, \ \pi_{k+1} ... \pi_n\)
    круг \( Q_0, \ Q_1, \ Q_2 ... Q_k, \ Q_{k+1} ... Q_n\)
    Площадь круга \( S_0, \ S_1, \ S_2 ... S_k, \ S_{k+1} ... S_n\)

    \( \forall k \ Z'_k, Z''_k ---\) цилиндры:
    \( Z'_k \subset T_k \subset Z''_k \)
    \( V(Z'_k) \leq V(T_k) \leq V(Z''_k) \)

    % wtf if U
    %finish the U thing
    \( Z' \subset T = U T_k \subset Z'' \)
    \( Z'lksjfg \)

    кусочно-цилиндрические тела
    \(\downarrow\)

    \( V(Z') \leq V(T) \leq V(Z'') \)

    Надо для \( \exists! V(t) \)

    Показать, что \( \forall \varepsilon V(Z'') - V(Z') < \varepsilon \)

    \(\begin{cases}
        V(Z') = \frac{b-a}{n} (S_0 + S_1 + ... + S_{n-1})
        V(Z'') = \frac{b-a}{n} (S_0 + S_1 + ... + S_{n})
    \end{cases} \Rightarrow V(Z'') - V(Z') = \frac{b-a}{n}(S_n - S_0) < \varepsilon\)

    \(N = [\frac{(b \cdot a)(S_n - S_0)}{\varepsilon}] + 1\)

    \(\Rightarrow \exists! V(T)\)

    %25:30
    \(2\) %line

    \(x \in [x_k; x_{k+1}]\) т.к. \(f(x) \uparrow S(x)\)

    %
    %
    %
    %

    \(\begin{cases}
        V(Z') = \sum{k = 0}{n - 1}S_k\frac{b-a}{n}
        V(Z') = \sum{k = 0}{n - 1}S_{k+1}\frac{b-a}{n}
    \end{cases} \textrm{знаем, что} S_k \leq S(x) \leq S_{k+1}\)
    тогда:
    \(V(Z') ---\) нижняя
    \(V(Z'') ---\) верхняя интегральная сумма для \(S(x)\) если $S(x)$ — кус.непр., то:
    \(\exists! I = \int_{a}^{b} S(x)dx \)

    \( V(Z') \leq I \leq V(Z'') \)
    \(1. + 2. \Rightarrow V(T) = \int_{a}^{b}S(x)dx \)
    \(3. S(x) = \pi f^2(x) \Rightarrow V(T) = \pi \int_a^b f^2(x)dx \)

    \textbf{Замечание.}
    Всё тоже самое для \(\forall f(x): f(x) \downarrow, f(x) --- \textrm{непр.} \Rightarrow \forall f(x): f(x) \geq 0, f(x) --— \textrm{непр.}, f(x) --- \textrm{кусочно монотонна}\)

    \textbf{Пример 1:}

    %last 5–10 minutes of lesson 1 on April 12, 2022



    %lesson 2
    % graph (start of lesson 2)
    \textbf{Пример 2:}
    \(V_{\textrm{шара}}; R\)
    \(x^2 + y^2 = R^2 \\ y = \sqrt{R^2 - x^2} \)

    \( V = \pi \int_{-R}^{R}(\sqrt{R^2 - x^2})^2 dx =  \pi \int_{-R}^{R} (R^2 - x^2) dx = 2\pi \int_{0}^{R} (R^2 - x^2) dx = \frac{4\pi R^3}{3}\)

    Забавное наблюдение:
    \(\pi R^2 \rightarrow 2 \pi R\)
    \(\frac{4\pi R^3}{3} \rightarrow 4\pi R^2\)

    \textbf{Замечание.} \(\pi \int_a^b f^2(x)dx ---\) это частный случай формулы объёма теа по площадям Ⅱ-ых сечений.

    %10:00
    
    %начинать её выложенный ролик начинать на 35 минуте




    \section*{Элементы Теории вероятности} 
    \subsection*{Введение. Виды событий} %п.1
    Событие (в житейном понятии) — это искон некоторого опыта или эксперимента.
    Виды событий: достоверные, невозможные, случайные.

    \textbf{Пример 1: Подбрасывание игральной кости}
    Достоверное событие — "число выпавших очков — натуральное число"
    Невозможное событие — "число выпавших очков равно 7"
    Возмоюное(случайное) событие — "число выпавших очков равно 5"
    
    \textbf{Пример 2: Подбрасывание камня над поверхностью Земли}
    Достоверно — камень упадёт на Землю
    Невозможно — камень не упадёт на Землю
    Возмоюно — канень упадёт на Землю через 10 метров

    \textbf{Теория вероятности} --- раздел математики, изучающий закономерности случайных событий.



    \subsection*{Вероятноство пространство} %п2
    \textbf{Определение:} Множество \(U = {X_1, X_2, ..., X_n}\), состоящее из конечного числа исходов данного опыта, называется \textbf{пространством элементарных событий}, если:
    \begin{enumerate}
        \item опыт может закончится лишь одним из перечисленных исходов;
        \item опыт не может закончится никаким другим исходом, кроме перечисленных.
    \end{enumerate}

    \textbf{Пример 1:} Подбрасывание игральной кости
    \begin{enumerate}
        \item \(U = {A_1, A_2, A_3, A_4, A_5, A_6}\), где $A_k$ --- выпадение $k$ очков.
        \item \(U = {B_0, B_1}\), где $B_0$ ––– выпадение чётного числа очков, $B_1$ --- выпадение нечётного числа очков;
        \item \(U = {C_1, C_2\), где $C_1$ ––– выпадение чётного числа очко
    \end{enumerate}

    %%% ДОДЕЛАТЬ фото на телефон 

    \textbf{Определение:}
    Вероятностным пространством называют конечное множество \(U = {X_1, X_2, ... , X_n}\), каждому элементу $X_k$ которого поставлено в соответствие неотрицательое число \(p(X_k) = p_k: p_1 + p_2 + ... + p_n = 1\).
    Обозначение: \((X_1, X_2, ... , X_n, p_1 + p_2 + ... + p_n)\). В дальнейшем будем для кратности писать $(U,P)$, где \(U = {X_1, X_2, ... , X_n}, \ P = {p_1, p_2, ... , p_n}\)

    \textbf{Пример 2:}
    \begin{enumerate}
        \item Подбрасывание симметричной монеты \(U = {Heads, Tales}, P = {\frac{1}{2}; \frac{1}{2}}\)
        \item Подбрасывание симметричной кости %%%%%%%
        \item Подбрасывание двух симметричных монет %%%%%%
    \end{enumerate}

    Равнвероятны %%%%%%%%
    

    \subsection*{Вероятность событий} %п3

    \(U = {A_1, A_2, A_3, A_4, A_5, A_6}, X ---\) выпало простое число, $Y$ --- число $>2$ и $<5$
    \textbf{Определение:}
    Событие при данном испытании называется любое подмножество $X$ множества исходов \(U = {X_1, X_2, ... X_n}\).
    В дальнейшем \(X_i \in X\) ——— благоприятные исходы, \(X_i \in U\) и $\notin X$ ——— неблагоприятное ихсоды.

    \textbf{Определение:}
    Вероятность события $X$ называется сумма вероятностей благоприятных ему элементарных исходов:
    
    \(P(X) = \sum_{i} P(X_i)\), где $X_i$ ——— благоприятные $X$ исходы.

    \textbf{Замечание:} Если работаем с вероятностным пространством, в котором все элементарные изходы равновероятны, то вероятность события есть:
    \( \frac{ \textrm{число благоприятных изходов} }{ \textrm{число всех изходов} }\)

    \textbf{Пример 1:}
    Найдём $P(X)$ и $P(Y)$, где \(X = {A_2, A_3, A_5}, Y = {A_3, A_4}\)

    \(P(X) = \)

    \textbf{Пример 2:} Что вероятнее всего выбросить при метании двух костей 7 очков или 8 очков.
    Исходы для первой кости \(A_1, A_2, ..., A_6\) для второй: \(B_1, B_2, ..., B_6\)
    Исходы при подбрасывании двух костей --- это пары \((A_i, B_j), i, j = 1, ... , 6.\) Кости между собой не скреплены, поэтому таких пар 36.
    7 очков:  \( (A_1, B_6), (A_2, B_5), (A_3, B_4), (A_4, B_3), (A_5, B_2), (A_6, B_1). P(7) = \frac{6}{36} = \frac{1}{6}.\)
    6 очков:  \( (A_2, B_6), (A_3, B_5), (A_4, B_4), (A_5, B_3), (A_6, B_2). P(8) = \frac{5}{36}.\)


    \textbf{Пример 3:} 
    В мешке лежат 33 жетона, помеченные буквами русского алфавита.
    Из него извлекают жетоны и записывают соответствующие буквы, 
    причем вынутые жетоны обратно не возвращаются. 
    Какова вероятность того, что при этом получиться слово "око"? слово "ар"?

    \(P(\textrm{"око"})=0\), так как вторую "о" не вытащим.

    "ар": всего исходов \(33*32=1056\), благоприятный - 1, по этому \(P(\textrm{"ар"})=\frac{1}{1056}\).

    \textbf{Пример 4:}
    Из мешка с 33 жетонами, помеченными буквами русского алфавита, вынимают 6 жетонов и записывают соответствующие буквы в порядке извлечения. Какова вероятность получить слово "Москва", если 1) жетоны возвращаются, 2) жетоны не возвращаются.
    \begin{enumerate}
    \item \(33 \cdot 33 \cdot 33 \cdot 33 \cdot 33 \cdot 33 \) ——— всего различных исходов, исходы равновероятны, благоприятны ——— 1. \(P(\textrm{"Москва"}) = \frac{1}{33^6}\)
    \item всего различных исходов \(33 \cdot 32 \cdot 31 \cdot 30 \cdot 29 \cdot 28 \) исходы равновероятны, благоприятный ——— 1. \(P(\textrm{"Москва"}) = \frac{1}{33 \cdot 32 \cdot 31 \cdot 30 \cdot 29 \cdot 28}\)
    \end{enumerate}

    \textbf{Пример 5:}
    Из квадратиков с буквами сложили слово "Миссисипи", после чего квадратики сложили в мешок и перемешали. 
    Какова вероятность, что после поочередного извлечения из мешка получиться тоже самое слово?

    Здесь равновероятными исходами являются появления любых перестановок с повторениями из 1-ой "м", 4-ех "и", 3-ех "с", 1-ой "п". 
    Таких исходов \( \frac{9!}{1! \cdot 4! \cdot 3! \cdot 1!} = \frac{5 \cdot 6 \cdot 7 \cdot 8 \cdot 9}{2 \cdot 3} = 10 \cdot 7 \cdot 4 \cdot 9 = 2520\)
    Благоприятный ——— 1. Поэтому \(P(X) = \frac{1}{2520}\)
    
    
    \textbf{Привер 6:}
    В мешке лежит 5 жетонов, помеченных буквами "а", "б", "в", "г", "д". 
    Из него 4 раза вынимают жетон, который после записи его буквы возвращается обратно. 
    Какова вероятность, что при этом ни одна буква не повториться дважды?

    Число равновероятных исходов \(5 \cdot 5 \cdot 5 \cdot 5 = 625\)
    
    Благоприятный исходы --- размещения без повторений из 5 по 4, их \(5 \cdot 4 \cdot 3 \cdot 2 = 120\)
    \(P(x)=\frac{120}{625} = \frac{24}{125} = 0,192\)

    
    \textbf{Пример 7:}
    При тех же условиях найдем вероятность того, что в полученной записи никакие двесоседние буквы не будут одинаковы. 
    
    Число всех равновероятных исходов по прежнему  \(5 \cdot 5 \cdot 5 \cdot 5 = 625\)
    Число благоприятных исходов \(5 \cdot 4 \cdot 4 \cdot 4. P(X) = \frac{64}{125} = 0,512\)

    
    \textbf{Пример 8:}
    Из мешка с 33 жетонами на которых написаны буквы русского алфавита, извлекаются 4 жетона, которые располагаются в алфавитном порядке. Какова вероятность того, что при этом получиться слово "ВИНТ"?
    Так жетоны располагаются в алфавитном порядке, то важно лишь то, какие буквы вынимаются, без учёта порядка. Общее число равновероятных исходов равно числу сочинений из 33 по 4:
    \( C_{33}^{4} = \frac{33!}{4! \cdot 29!} = 40920 \)
    Благоприятный ——— 1. \(P(X) = \frac{1}{40920}\)

    
    \textbf{Пример 9:}
    Карточки на которых написанны буквы слова "обороноспособность" располагаются произвольным образом. 
    Какова вероятность, что все 7 букв "о" идут подряд?
    
    \textbf{1 Способ:}
    Если равновероятностные исходы --- перестановки с повторениями из букв слова. 
    В слове "о" --- 7 шт., "с" --- 3 шт., "б" --- 2 шт., "р", "п", "т", "ь" --- по 1 шт. (всего 18 букв).
    Всего исходов \(\frac{18!}{7!*3!*2!*2!}\)
    Благоприятные исходы. Семь "о" подряд может начинаться с 1-ой позиции, со 2-ой, \cdots, c 12-ой, то есть 12 вариантов, а на остальные 11 мест перестановки с повторениями 11 оставшихся букв. 
    Число благоприятных исходов равно \(\frac{12*11!}{3!*2!*2!}\)
    
    \(P(X)=\frac{12*11!}{3!*2!*2!}*\frac{7!*3!*2!*2!}{18!}=\frac{12*11!*7!}{18!}=\frac{1}{2652}\)
    
    \textbf{2 Способ:}
    Можно понять, что существенно лишь располижение букв "о".
    Всего исходов: выбор 7 мест из 18-ти для букв "0"
    \(C_{18}^7=\frac{18!}{7! \cdot 11!}\)

    Благоприятных исходов - 12 \(P(X)=\frac{12 \cdot 11! \cdot 7!}{18!}=\frac{1}{2652}\)
    
    \textbf{Пример 10:}
    Из мешка с 33 жетонами, на которых написаны буквы русского алфавита извлекают 4 жетона, 
    причём каждый жетон после извечения и записи его буквы возвращается в мешок. 
    Какова вероятность, что из полученных таким образом букв можно будет получить слово 1) "март" 2) "мама"


    
    %типа начало 13 лекции%
    
    \subsection{Алгебра событий}
    \textbf{Определение 1.} Событие, которому не благоприятен ни один из возможных исходов, называется \textit{невозможным}: \(\varnothing\)
    Событие, которому благоприятен любой исход испытания, называется \textit{достоверным}: $U$

    \textbf{Определение 2.} \textit{Обьединением (суммой}) событий $X$ и $Y$ называется событие \(X \bigcup Y (X + Y)\), которому  благоприятны все исходы, благоприятные хотя бы одному из событий.
    
    \textbf{Пример 1:} При бросании двух костей объединением двух событий "выпало чётное число очков" и "выпало простое число очков" будет событие \(X \bigcup Y\) = {2,3,4,5,6,7,8,9,10,11,12} = {число выпавших очков \( \ne \) 9}.
    
    \textbf{Пример 2:} Если два зенитных орудия стреляют по одному и тому же самолёту, 
    то объединением событий "первое орудие поразило самолёт" 
    и "второе орудие поразило самолёт" будет "самолёт сбит".

    \textbf{Определение 3.} \textit{Пересечением} \(X \bigcup Y (XY)\) событий называют событие, которому блягорпиятны лишь исходы, одновременно благоприятные и для $X$, и для $Y$. 
    %

    \textbf{Определение 4.} Два события называют \textit{несовместным}, если их пересечением является невозможное событие.

    \textbf{Пример 3:} Подбрясывание игральной кости.
    "Число выброшенных очков кратно 3" и "число выброшенных очков даёт при делении на 3 в остатке 1" ——— несовместные события

    \textbf{Определение 5.} Если любые два события из множества \(X_1,X_2,\cdots ,X_n}\) несовместны, 
    то эти события называются \textit{попарно несовместными}.
    
    \textbf{Пример 3':}
    Попарно несовместны "число выброшенных очков кратно 3", "число выброшеных очков даёт при делении на 3 остаток 1",
    "число несовместных очков при делении на 3 даёт в остатке 2".

    \textbf{Пример 3":}
    события \(X = \{\textrm{выпало простое число}\}, Y = \{\textrm{выпало чётное число очков}\}, Z = \{\textrm{выпало число очков Ю}\}\) \textit{несовместны в совокупности},
    но не являются попарно несовместными.
    
    Так, например: 2 - простое и чётное, 5 - простое и >3.
    Диаграмма Эйлера-Венна для множества исходов этих событий имеет вид:
    %прикрепить диаграмму 
    %доделать слайд%

    Выбор другого множества элементарных исходов.
    
    \textbf{Пример 4:} При бросании одной кости события "выпало чётное число очков" и "выпало нечётное число очков" образуют полную систему исходов, причём они равновероятны. \(X_1 = \{A_2, A_4, A_6\}\), \(X_2 = \{A_1, A_3, A_5\}\), \(P(X_1) = P(X_2) = 3 * 1 / 6 = 1 / 2\).
    \textbf{Определение 6.} События $X$ и $Y$ называют \textit{противоположными}, если любой исход испытания благоприятен одному и только одному из них. Если \(Y\) противоположно \(X\), то \(Y = \lnot X\)
     
    \textbf{Пример 5:} Являются ли противоположными события?
    В случае подпбрасывания игральной кости.
    \begin{enumerate}
        \item \(X = {\textrm{первое орудие поразило самолёт}}, Y = {\textrm{второе орудие поразило самолёт}}, \)
        \item \(Ч = \{\textrm{число очков < 3}\}, Y = \{\textrm{число очков > 3}\}\)
    \end{enumerate}
    \textbf{Определение 7.} Событие $Y$  называется \textit{следствием} события $X$, если любой исход $X$

    
    \textbf{Итак!}
    Мы ввели основные операции над событиями. С их помощью можно определить другие операции.
    Например, событие \(X \upcup \lnot Y\)

    %% попросить у неё этот слайд...
    
    \textbf{Важно!}
    Поскольку операции над событиями сводятся к соответствующим операциям над множествами благоприятных им исходов, то все утверждения алгебры множеств остаются справедливы.
    Операции обьединения и пересечения %закончить%
    
    
    \subsection{Теоремы сложения}
    \textbf{Теорема 1.} Если ... объединения \( P(A \cup B) = P(A) + P(B)\)
    \(\uparrow\) Обозначим исходы благоприятные для события $А$ через \(a_1, a_2, ..., a_m,\) а исходы, благоприятные для $B$
    
    \textbf{Пример 1:}
    Стрелок стрелярет в мишень. Вероятность ввыбить 10 очков  равно 0,3, ена вероятность выбить 9 очков равна  Вероятность выбить 10 очков равна 0,3, а вероятность выбить 9 очков равна 0,6. Чему равна вероятность выбить не менее 9 очков?
    
    Не менее 9 очков, значит 9 или 10 очков.
    %дописать%


    \textbf{Пример 2.} В цехе работает несколько станков.
    Вероятность того, что за смену потребует наладки     \textbf{Утверждение.} Если события \(A_1, A_2, ... A_n\) попарно несовместны, то событие \(A_1 \bigcup A_2 \bigcup ... \bigcup A_n-1\) несовместно с событием $A_n$.
   
    \((A_1 \bigcup A_2 \bigcup ... \bigcup A_n-1) \bigcup \) %дописать%
ема 2.} Для любого события $A$ имеем: \(P( \ove{A} ) = 1 - P(A)\)
    Чему равна вероятность выбить не менее 9 очков?

    Не менее 9 очков, значи 9 очков или 10 очков.

    События \(A = \{\textrm{выбить 10 очков}\}\) и \(B = \{\textrm{выбить 9 очков}\}\) несовместны, 
    по этому \(P(A \cup B) = 0,3 + 0,6 = 0,9)\)

    
    \textbf{Утверждение.} Если события \(A_1, A_2 ... A_n\) попарно несовместны, то событие \( A_1 \cup A_2 \cup ... \cup A_{n-1} \) несовместно с событием \(A_n\).

    \( \uparrow (A_1 \cup A_2 \cup ... \cup A_{n-1}) \cap A_n = (A_1 \cap A_n) \cup (A_2 \cap A_n) \cup ... \cup (A_{n-1} \cap A_n) = \varnothing\)

    %дописать%

    \textbf{Теорема 2.} Для любого события $A$ имеем: \(P(\lnot A) = 1 - P(A)\)
    \(\uparrow\) Достаточно вспомнить, что \(A \cup \lnot A = U, A \cap \lnot A = \varnothing, P(U) = 1\), тогда по Т.1: \( 1 = P(U) = P(A) + P(\lnot A)\)

    \textbf{Пример 3:} Берётся наудачу трёхзначное натуральное число. Какова вероятность, что хотя бы 2 его цифры совпадают?
    Опыт состоит в том, что наудачу выбираются натуральное число от 100 до 999 и анализируется, есть ли у него одинаковые цифры или нет.
    События "влэ"

    %доделат%


    \textbf{Теорема 3.} Для любых двух событий справедливо равенство: \( P(A \cup B) = P(A) + P(B) - P(A \cap B)\)
    $\uparrow$ Заметим, что: \( A = A \cap U = A \cap (B \cup \lnot B) = (A \cap B) \cup (A \cap \lnot B) \)
    Причём: \( (A \cap B) \cup (A \cap \lnot B) = A \cap (B \cup \lnot B) = A \cap \varnothing = \varnothing \)
    Аналогично: \( B = (A \cap B) \cup (\lnot A \cap B) \)
    Поэтому:
    
    Кроме того: \( \begin{abstract}

    \end{abstract} \rightarrow 
        \)

    \textbf{Замечание}
    
     \end{document}