
\documentclass{article}

\usepackage[utf8x]{inputenc}
\usepackage[english,russian]{babel}
\usepackage{cmap}
\usepackage{commath}
\usepackage{amsmath}
\usepackage{amsfonts}
\usepackage{mathtools}
\usepackage{amssymb}
\usepackage{parskip}
\usepackage{titling}
\usepackage{color}
\usepackage{hyperref}
\usepackage{cancel}
\usepackage{enumerate}
\usepackage{multicol}
\usepackage{graphicx}
\usepackage[font=small,labelfont=bf]{caption}
\usepackage[a4paper, left=2.5cm, right=1.5cm, top=2.5cm, bottom=2.5cm]{geometry}

\graphicspath{ {./images/} }
\setlength{\droptitle}{-3cm}
\hypersetup{ colorlinks=true, linktoc=all, linkcolor=blue }
\pagenumbering{arabic}

\begin{document}

    \( \begin{pmatrix}
        x\\
        y\\
        z
    \end{pmatrix} = \overrightarrow{X} \)

    \( \begin{pmatrix}
        d_1\\
        d_2\\
        d_3
    \end{pmatrix} = \overrightarrow{D} \)

    \( A \cdot X = D (\star\star) \)

    \( A^{-1} \cdot A = E \)

    \( X = A^{-1}AX = A^{-1}D \)

    \textbf{Определение.} Определитель \(\det A = \abs{A}\).
    
    \(\begin{vmatrix}
        a_{11} & a_{12} & a_{13}\\
        a_{21} & a_{22} & a_{23}\\
        a_{31} & a_{32} & a_{33}
    \end{vmatrix} = 
    a_{11}a_{22}a_{33} + a_{12}a_{23}a_{31} + a_{21}a_{13}a_{32} - a_{13}a_{22}a_{31} - a_{12}a_{21}a_{33} - a_{23}a_{32}a_{11} = a_{11} \begin{vmatrix}
        a_{22} & a_{23}\\
        a_{32} & a_{33}
    \end{vmatrix} - a_{12} \begin{vmatrix}
        a_{21} & a_{23}\\
        a_{31} & a_{33}
    \end{vmatrix} + a_{13} \begin{vmatrix}
        a_{21} & a_{22}\\
        a_{31} & a_{32}
    \end{vmatrix}\)

    \(\begin{vmatrix}
        a_{11} & a_{12}\\
        a_{21} & a_{22}
    \end{vmatrix} = a_{11}a_{22} - a_{12}a_{21} \star\star\)

    \textbf{Теорема.} Если \( \det A \neq 0 \), то система имеет единственное решение.

    \( \det A = 0 \Rightarrow \)
    \begin{enumerate}
        \item \( \exists \infty \)
        \item \( \emptyset \)
    \end{enumerate}

    \( \begin{cases}
        a_{11}x + a_{12}y = d_1\\
        a_{21}x + a_{22}y = d_2
    \end{cases} \)
    
    \( \frac{a_{11}}{a_{21}} \neq \frac{a_{12}}{a_{22}} \exists! \)

    \subsubsection{Расстояние от точки до плоскости}

    \textbf{Дано:} 
    
    \( \alpha: ax + by + cz + d = 0 \)
    
    \( \rho (M_0, \alpha) = ?\)

    % TODO IMG 22:10
    \(M_0(x_0, y_0, z_0),\ n_\alpha(a, b, c)\)

    \( M_1 (x_1, y_1, z_1) \), где \( x_1 - ?, y_1 - ?, z_1 - ? \)
    
    \( \overrightarrow{M_0M_1} = (x_1 - x_0, y_1 - y_0, z_1 - z_0) \)

    \( \overrightarrow{M_0M_1} = \lambda\overrightarrow{n_\alpha} \Rightarrow \frac{x_1 - x_0}{a} = \frac{y_1 - y_0}{b} = \frac{z_1 - z_0}{c} = \lambda \)
    
    \[\Downarrow\]
    
    \begin{equation}\label{eq:*}
        \begin{cases}
            x_1 - x_0 = \lambda a\\
            y_1 - y_0 = \lambda b\\
            z_1 - z_0 = \lambda c\\
        \end{cases}
    \end{equation}

    \( M_1 \in \alpha \Rightarrow a(x_0 \lambda a) + b(y_0 \lambda b) + c(z_0 \lambda c) + d = 0 \)

    \( \lambda (a^2 + b^2 + c^2) = -(ax_0 + by_0 + cz_0 + d) \)
    
    \(\lambda = -\frac{ax_0 + by_0 + cz_0}{a^2 + b^2 + c^2}\)

    \( \rho(M_0; \alpha) = \abs{\overrightarrow{M_0M_1}} = \sqrt{(x_1 - x_0)^2 + (y_1 - y_0)^2 + (z_1 - z_0)^2} \stackrel{\ref{eq:*}}{=} \sqrt{\lambda^2a^2 + \lambda^2b^2 + \lambda^2c^2} = \abs{\lambda}\sqrt{a^2+b^2+c^2} = \frac{\abs{ax_0 + by_0 + cz_0}}{\sqrt{a^2 + b^2 + c^2}} \downarrow\)

    \( M_0(x_0, y_0) \)
    % TODO IMG 32:51

    \( ax + by + d = 0 \)

    \( \rho(M_0, l) = \frac{\abs{ax_0 + by_0 + d}}{\sqrt{a^2 + b^2}} \)

    \subsubsection{Две плоскости}

    \( \alpha: a_1x + b_1y + c_1z + d_1 = 0 \)
    \( \beta: a_2x + b_2y + c_2z + d_2 = 0 \)

    \begin{enumerate}
        \item \( \alpha || \beta \Rightarrow n_\alpha || n_\beta \Rightarrow \frac{a_1}{a_2} = \frac{b_1}{b_2} = \frac{c_1}{c_2} = k \)
        
        \( \begin{cases}
            a_1 = ka_2\\
            b_1 = kb_2\\
            c_1 = kc_2
        \end{cases}\)

        \item \( \alpha \perp \beta \Rightarrow n_\alpha \perp n_\beta \Rightarrow a_1a_2 + b_1b_2 + c_1c_2 = 0 \)
        
        % TODO IMG 38:40

        \item \( \angle(\alpha; \beta) = \angle(n_\alpha; n_\beta) = \angle \phi \leq 90^\circ  \)

        \(\cos(\alpha; \beta) = \begin{cases}
            \cos\angle(n_\alpha; n_\beta),\ \cos \phi \geq 0\\
            -\cos\angle(n_\alpha; n_\beta),\ \cos \angle(n_\alpha; n_\beta) < 0
        \end{cases} = \abs{\cos\angle(n_\alpha; n_\beta)} = \frac{\abs{a_1a_2 + b_1b_2 + c_1c_2}}{\sqrt{a_1^2 + b_1^2 + c_1^2}\sqrt{a_2^2 + b_2^2 + c_2^2}} \)
    \end{enumerate}

    \subsubsection{Прямая в пространстве. Способы задания.} 

    \paragraph*{Первый способ. Две точки.}\mbox{}\\
    
    \( A_1(x_1, y_1, z_1) \)

    \( A_2(x_2, y_2, z_2) \)

    Уравнение прямой?

    \( \overrightarrow{A_1A_2} = (x_2 - x_1, y_2 - y_1, z_2 - z_1) \)

    % TODO IMG 09:50

    \( B(x; y; z) \)

    \( \overrightarrow{OB} = (x; y; z) = \overrightarrow{OA_1} + \overrightarrow{A_1B} = \overrightarrow{OA_1} + t\overrightarrow{A_1A_2}\)

    \( \overrightarrow{OB} = \overrightarrow{OA_1} + t\overrightarrow{A_1A_2} \)

    \( \overrightarrow{r} = \overrightarrow{r_{A1}} + t\overrightarrow{A_1A_2} \)

    \( \overrightarrow{A_1B} = t\overrightarrow{A_1A_2} \)

    \(\overrightarrow{A_1A_2}\) --- направляющий вектор.

    \( \begin{cases}
        x - x_1 = t(x_2 - x_1)\\ 
        y - y_1 = t(y_2 - y_1)\\
        z - z_1 = t(z_2 - z_1)
    \end{cases} \)

    \( \begin{cases}
        x = x_1 + t(x_2 - x_1)\\ 
        y = y_1 + t(y_2 - y_1)\ t \in \mathbb{R}\\
        z = z_1 + t(z_2 - z_1)
    \end{cases} \)

    Это параметрическое задание прямой.

    \(\begin{cases}
        x = x_1 + t\tau_1\\
        y = y_1 + t\tau_2\ t \in \mathbb{R}\\
        z = z_1 + t\tau_3
    \end{cases}\)

    Каноническое уравнение прямой: 
    
    \(\frac{x-x_1}{\tau_1} = \frac{y-y_1}{\tau_2} = \frac{z-z_1}{\tau_3} = t\)

    \paragraph*{Второй способ. Пересечение двух плоскостей.}\mbox{}\\
    
    \( \begin{cases}
        a_1x + b_1y + c_1z + d_1 = 0\\
        a_2x + b_2y + c_2z + d_2 = 0
    \end{cases} \)

    \textbf{Пример.} 

    \( \begin{cases}
        a_1x + b_1y + c_1z + d_1 = 0\\
        z = 0
    \end{cases} \)

    % TODO IMG 22:35

    \(\begin{cases}
        x = t\ t \in \mathbb{R}\\
        b_1y + c_1z = -d_1 - a_1t\\
        b_2y + c_2z = -d_2 - a_2t
    \end{cases}\\
    \begin{cases}
        x = t\\
        y = \varphi(t)\\
        z = \psi(t)
    \end{cases}\)

    \paragraph{Угол между прямыми}\mbox{}\\

    \( \begin{array}{l}
        l_1 \Rightarrow (\tau_1, \tau_2, \tau_3) = \tau\\
        l_2 \Rightarrow (\nu_1, \nu_2, \nu_3) = \nu 
    \end{array} \) направляющие

    \(\angle(l_1, l_2) = \begin{cases}
        \angle(\overrightarrow{\tau}; \overrightarrow{\nu});\ \angle(\overrightarrow{\tau}; \overrightarrow{\nu}) \leq 90^\circ\\
        180^\circ - \angle(\overrightarrow{\tau}; \overrightarrow{\nu});\ \angle(\overrightarrow{\tau}; \overrightarrow{\nu}) > 90^\circ
    \end{cases}\)

    \( \cos\angle(l_1, l_2) = \abs{\cos\angle(\overrightarrow{\tau}; \overrightarrow{\nu})} = \frac{\abs{\tau_1\nu_1 + \tau_2\nu_2 + \tau_3\nu_3}}{\sqrt{\tau_1^2 + \tau_2^2 + \tau_3^2}\sqrt{\nu_1^2 + \nu_2^2 + \nu_3^2}} \)

    \paragraph{Угол между прямой и плоскостью.}\mbox{}\\

    \( l: \Rightarrow \overrightarrow{\tau} = (\tau_1; \tau_2; \tau_3) \)
    \( \alpha: \Rightarrow \overrightarrow{n_\alpha} = (a; b; c) \)

    % TODO IMG 33:18

    \( \angle\psi = \angle(\alpha; l) = \begin{cases}
        \frac{\pi}{2} - \angle(\overrightarrow{n_\alpha}, \overrightarrow{\tau}) = \frac{\pi}{2} - \varphi;\ \varphi \leq \frac{\pi}{2}\\
        \angle(\overrightarrow{n_\alpha}, \overrightarrow{\tau}) - \frac{\pi}{2} = \varphi - \frac{\pi}{2};\ \varphi > \frac{\pi}{2}
    \end{cases}\)

    \( \sin \psi = \begin{cases}
        \cos \phi,\ \cos \phi \geq 0\\
        -\cos \phi,\ \cos \phi < 0
    \end{cases} = \abs{\cos \phi} \Rightarrow\\ \Rightarrow \sin\angle(l; \alpha) = \frac{\abs{a\tau_1 + b\tau_2 + c\tau_3}}{\sqrt{a^2 + b^2 + c^2}\sqrt{\tau_1^2 + \tau_2^2 + \tau_3^2}} \)

    \section{Комплексные числа} 

    (Лекция по Яковлеву)

    Рассмотрим упорядоченные пары:

    \( (a, b); a, b \in \mathbb{R}  \)

    \( (1, 0), (1, \sqrt{2}) ... \)

    \( (a, b) \neq (b, a) \)

    \( (a_1, b_1) = (a_2, b_2) \Leftrightarrow \begin{cases}
        a_1 = a_2\\
        b_1 = b_2
    \end{cases} \)

    \subsection{Операции над парами}
    
    \subsubsection{Сложение}

    \( (a_1, b_1) + (a_2, b_2) \stackrel{df}{=} (a_1 + a_2, b_1 + b_2) \)

    \subsubsection{Умножение}

    \( (a_1, b_1) \cdot (a_2, b_2) = (a_1a_2-b_1b_2; a_1b_2 + b_1a_2)\)

    \textbf{Определение.} Множество упорядоченных пар, с введенными операциями сложения и умножения, назовём множеством комплексных чисел.

    Комплексные числа обозначаются \( z, w \) и другими маленькими буквами латинского алфавита.

    \subsection{Свойства сложения и умножения}

    \begin{enumerate}
        \item \( z_1 + z_2 = z_2 + z_1 \)
        \item \( (z_1 + z_2) + z_3 = z_1 + (z_2 + z_3) \)
        \item \(z\) --- разность \(z_2\) и \(z_1\), если \(z+z_1 = z_2\)
        
        \(z = (a, b);\ z_1 = (a_1, b_1);\ z_2 = (a_2, b_2)\)

        \(z + z_1 = (a+a_1, b + b_1) = (a_2, b_2)\)

        \( \Updownarrow \)

        \( \begin{cases}
            a + a_1 = a_2\\
            b + b_1 = b_2
        \end{cases} \Leftrightarrow \begin{cases}
            a = a_2 - a_1\\
            b = b_2 - b_1
        \end{cases} \)
        
        \(z = z_2 - z_1 = (a_2 - a_1, b_2 - b_1)\)

        \item \( 0 = (0, 0) \)
        \item \(z_1z_2 = z_2z_1\)
        \item \((z_1z_2)z_3 = z_1(z_2z_3)\)
        \item \( z:\ z_1 \cdot z = z_2, z_1 \neq 0 \)
        
        \( z = \frac{z_2}{z_1} \)

        \( (a_1, b_1) \cdot (a, b) = (a_2, b_2) \)

        \( (a_1a - b_1b, a_1b + b_1a) = (a_2, b_2) \Leftrightarrow
        \begin{cases}
            a_1a - b_1b = a_2\\
            a_1b - b_1a = b_2    
        \end{cases} \Leftrightarrow \begin{cases}
            a_1a-b_1b=a_2\\ b_1a+a_1b=b_2
        \end{cases}\)

        \(a = \frac{a_1a_2+b_1b_2}{a_1^2+b_1^2} \)    
        
        \(b = \frac{a_1b_2-b_1a_2}{a_1^2+b_1^2}\)    

        \( z=\frac{z_2}{z_1} = (\frac{a_1a_2 + b_1b_2}{a_1^2 + b_1^2}, \frac{a_1b_2 - b_1a_2}{a_1^2 + b_1^2}) \)
    \end{enumerate}

\end{document}
