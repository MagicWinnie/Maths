\documentclass{article}

\usepackage[utf8x]{inputenc}
\usepackage[english,russian]{babel}
\usepackage{cmap}
\usepackage{commath}
\usepackage{amsmath}
\usepackage{amsfonts}
\usepackage{mathtools}
\usepackage{amssymb}
\usepackage{parskip}
\usepackage{titling}
\usepackage{color}
\usepackage{hyperref}
\usepackage{cancel}
\usepackage{enumerate}
\usepackage{multicol}
\usepackage{graphicx}
\usepackage{docmute}
\usepackage[font=small,labelfont=bf]{caption}
\usepackage[a4paper, left=2.5cm, right=1.5cm, top=2.5cm, bottom=2.5cm]{geometry}

\graphicspath{ {./images/} }
\setlength{\droptitle}{-3cm}
\hypersetup{ colorlinks=true, linktoc=all, linkcolor=blue }
\pagenumbering{arabic}

\begin{document}
    % 26.04.2022
    \subsection{}

    \textbf{Условная вероятность}
    Пусть события \(A \subset U, B \subset U \)
    Имеет место классическое определение вероятности, т.е. вероятность каждого элементарного события равна \(\fraq{1}{n}\),
где $n$ ——— количество элементарных событий.
Известно, что событие $B$ произошло.
\textit{Какова вероятность того, что произошло событие $A$?}
\begin{enumerate}
    \item So
\end{enumerate}
%доделать слайд 20 ⬆️

\textbf{Определение.} Число, выражающее вероятность события $A$ при условии, что произошло событие $B$, называется условной вероятностью события $A$ относительно события $B$ и обозначается \(P(A|B)\)
\textbf{Утверждение.} Показали, что \(P(A|B) = \frac{P(A \cap B)}{P(B)} \ (1)\)
\textbf{Замечание.}
\begin{enumerate}
    \item Условная вероятность события $A$ при условии, что произошло событие $B$, имеющее нулевую вероятность, не определена.
    \item Формулу $(1)$ полезно записывать в виде: \(P(A \cap B) = P(A|B) \cdot P(B)\) \textif{формула умножения} \(P(A \cap B) = P(B|A) \cdot P(A)\)
\end{enumerate}
\textbf{Пример 1.} Нерадивый ученик выучил один из семи экзаминационных билетов. Сравним вероятность того, что он сдаст экзамен, если пойдёт его сдавать первым или вторым.

$A$ - ученик сдал экзамен. Сдать он может, если вытащил билет, который знает.\\
Если он войдёт первым, то \(P(A)=1/7\).\\
Если же ученик идёт вторым, то может сдать экзамен только если его "счастливый" билет не взял первый сдающий.\\
Пусть $B$ - первый сдающий не взял "счастливый" билет.\\
Тогда нам надо найти \(P(A \cap B) = P(A|B) \cdot P(B)\) (у нас \(A \subset B\), поэтому \(A = A \cap B\))


Итак, \(P(A \cap B) = P(B) \cdot P(A|B) = (\frac{6}{7}) \cdot (\frac{1}{6}) = \frac{1}{7}\)

%доделать слайд 21 ⬆ Done
    
\textbf{Пример 2}
В ящике лежат 8 белых и 12 чёрных шаров. Подряд вытаскивают два шара.
Найти вероятность того, что оба извлечённых шара будут белыми.

\textbf{Способ 1.} 
Если в качестве пространства элементарных событий рассмотреть все возможные пары шаров (с учётом порядка), то можно найти искомую вероятность по формуле классической вероятности.
Всего исходов \(20 \cdot 19\). Благоприятных исходов \(8 \cdot 7\). Тогда, \(P(б,б)=(8 \cdot 7)/(20 \cdot 19)=14/95\).

\textbf{Способ 2.}
$А$ — первый извлечённый шар белый, $B$ - второй извлечённый шар белый. 

Нас интересует \(P(A \cap B) = P(B|A)*P(A) = P(A)*P(B|A) = (\frac{8}{20})(\frac{7}{19}) = \frac{14}{95}\)

\subsection{Независимые случайные события}
\textbf{Определение 1.} Событие $A$ называется независимом от события $B$, если \(P(A|B) = P(A)\)

\textbf{Утверждение.} Если событие $А$ независимо от события $B$, то событие $B$ независимо от события $A$.
\( \uparrow \) По определению независимости события $A$ от события $B$:
\( P(A) = P(A|B) = \frac{P(A \cap B)}{P(B)} \Rightarrow P(A \cap B) = P(A) \cdot P(B) \Rightarrow P(B) = \frac{P(A \cap B)}{P(A)} = P(B|A) \downarrow\)

\textbf{Определение 1'.} События $A$ и $B$ из одного и того же вероятностного пространства называются независимыми, если \(P(A \bigcap B) = P(A)*P(B)\). 

\textbf{Замечание:} \textit{Не путайте несовместные события с независимыми!}

\textbf{Пример 1.} Докажем, что при бросании кости события $A$ ——— выпало чётное число очков и $B$ ——— число выпавших очков делится на 3 являются независимыми.

Действительно \(P(A) = \frac{1}{2}, P(B) = \frac{2}{6} = \frac{1}{3}, P(A \cap B) = P(6) = \frac{1}{6} = \frac{1}{2} \cdot \frac{1}{3}\), т.е. независимы. 

\textbf{Утверждение} Если события А и Б независимы, то независимы события %A и неБ     неА и Б    неА и неБ
\(\uparrow\) Действительно, по правила алгебры событий имеем:
\( A = (A \cap B) \cup (A \cap \not B)\) причём \( (A \cap B) \cap (A \cap \not B) = \varnothing\)
Тогда \( P(A) = P(A \cap B) + P(A \cap \not B) \\ P(A \cap \not B) = P(A) - P(A)P(B) = P(A)(1 - P(B)) = P(A)P(\not B)\)
Аналогично доказывается независимость двух других пар событий. \(\downarrow\)


\textbf{Пример 2.} Два зенитных орудия стреляют одновременно и независим друг от друга по самолётам. Самолёт сбит, если в него попал хотя бы один снаряд. Какова вероятность сбить самолёт, если вероятность попадания первого орудия равна 0.8, а второго - 0.75?

Рассмотрим события
$A$ - "самолёт сбит первым орудием" и $B$ - "самолёт сбит вторым орудием".
Нас интересует \(P(A \cup B)\) с учётом того, что рассматриваемые события независимы.

\(P(A \cup B) = P(A) + P(B) - P(A \cap B) = P(A) + P(B) - P(A)P(B) = 0,8 + 0,75 - 0,8 \cdot 0,75 = 1,55 - \frac{4}{5} \cdot \frac{3}{4} = 1,55 - 0,6 = 0,95\)
%доделать слайд 23 ⬆️. DONE
\textbf{Определение 2.} События $A$, $B$, $C$,... называются \textit{независимыми}, если для любого подмножества этих событий вероятность их пересечения равноа произведению их вероятностей\dots

\textbf{Пример 3.} Четыре охотника стреляют одновременно и независимо друг от друга по зайцу. 
Заяц подстрелен, если попал хотя бы один охотник. Какова вероятность подстрелить зайца, если вероятность попадания каждого охотника \(\frac{2}{3}\)

Пронумеруем охотников и рассмотрим события $A_k$ - "попал $k$-ый охотник",\\
\(P(A_k) = 2/3\). Требуется найти \(P(A_1 \cup A_2 \cup A_3 \cup A_4)\).\\
Будем искать, используя вероятность противоположного события. 

\( P(A_1 \cup A_2 \cup A_3 \cup A_4) = 1 - \lnot P(A_1 \cup A_2 \cup A_3 \cup A_4) = 1 - P(\lnot A_1 \cap \lnot A_2 \cap \lnot A_3 \cap \lnot A_4) = 1 - P(\lnot A_1)P(\lnot A_2)P(\lnot A_3)P(\lnot A_4) = 1 - (\frac{1}{3})^4 = \frac{80}{81} = 0,988 \)

Здесь мы воспользовались тем, что при замене независимых событий противоположными, независимость событий не нарушается.


\textbf{Пример 4.} Событие $A$ может произойти в опыте с вероятностью $p$. Опыт повторили \textif{независимым} образом $n$ раз. Какова вероятность, что при этом событие $A$ произойдет хоть один раз.

Рассмотрим события \(A_k\) — событие $A$ произошло при $k$-ом повторении опыта. События \(A_k\) независимы (опыт повторяется независимым образом),
\(P(A_k) = p\) для \(k = 1,..., n.\)
Нас интересует событие "событие $A$ произошло хотя бы один раз", т.е. событие \(A_1 \cup A_2 \cup ... \cup A_n = U^n_{k = 1}A_k\)

Заметив, что \(\lnot A_k\) независимы и \(P(\lnot A_k) = 1 - p = q\), получаем:

\[P(\cup_{k=1}^{n} A_k) = 1 - P(\lnot \cup_{k=1}^{n} A_k) = 1 - P(\cap_{k=1}^{n} )\]

%написать первое замечание% (я его вставил. отредачь. выдели латинскые буквы)
\textbf{Замечание:} Интересно отметить, \( \lim_{n \to \infty} P(A) = \lim_{n \to \infty}(1 - q)= 1\) , поскольку \(0<q<1\), т.е. при достаточно большом числе повторений опыта событие $A$ произойдет почти наверняка (с вероятностью как угодно близкой к 1) хоть один раз. Про события, которые происходят почти наверняка, принято говорить, что они практически достоверны.
\textbf{Замечание:} Из попарной независимости событий не следует их независимость в совокупности. 

%доделать слайд 26 ⬆️


\subsection{Формула полной вероятности. Формула Байеса.}

\textbf{Пример 1}
Пусть имеется $a$ белых и $b$ черных мешков, причём в каждом белом мешке лежит $x$ красных и $y$ синих шаров, 
а в каждом черном мешке ——— лежит $u$ красных и $v$ синих шаров. 
Сначала случайным образом выбирают один мешок, 
а потом из него вынимают шар. Найти вероятности \(P(W), P(B), P(R|W), P(R|B), P(W \cap R), P(R).\)

Очевидно, что \(P(W) = \frac{a}{a + b}\), \(P(B) = \frac{b}{a + b}\), \(P(R|W) = \frac{x}{x + y}\), \(P(R|B) = \frac{u}{u + v}\). 

Событие \(W \cap R\) состоит в том, что выбран белый мешок, а из него извлечен красный шар. По формуле умножения

\(P(W \cap R) = P(R|W) \cdot P(W) = \frac{x}{x + y} \cdot \frac{a}{a + b}\).

Заметим, что событие $R$ ——— вынут красный шар является объединением непересекающихся (несовместных) событий \(W \cap R\) и \(B \cap R\).
Поэтому по теореме сложения имеем:

\(P(R) = P(W \cap R) + P(B \cap R) = \frac{x}{x + y} \cdot \frac{a}{a + b} + \frac{u}{u + v} \cdot \frac{b}{a + b}\).
%доделать слайд 27 ⬆️ Done


\textbf{Теорема 1} Формула полной вероятности.

Пусть вероятностное пространство $U$ представлено в виде объединения попарно несовместных событий \(X_1, X_2,..., X_n: U = X_1 \cup X_2 \cup ... \cup X_n\), где \(X_i \cap X_j = \varnothing\) при \(i \not = j\). 
Тогда для любого события A верно равенство:

\(P(A) = P(A|X_1) \cdot P(X_1) + P(A|X_2) \cdot P(X_2) + ... + P(A|X_n) \cdot P(X_n)\).

\(\uparrow\)
\begin{enumerate}
    \item в силу дистрибутивности операции пересечения событий относительно объединения имеем:
    
    \(A = A \cap U = A \cup (X_1 \cup X_2 \cup ... \cup X_n) = (A \cap X_1) \cup (A \cap X_2) \cup ... \cup (A \cap X_n)\).

    \item при этом из \(X_i \cap X_j = \varnothing\) следует \( (A \cap X_i) \cap (A \cap X_j) = A \cap X_i \cap X_j = \varnothing\).
    \item из 1) и 2) событие $A$ представимо в виде объявления попарно несовместимых событий. Его вероятность по теореме сложения равна %%%
    \item осталось... %%%
\end{enumerate}

%доделать слайд 28 ⬆️


\textbf{Следствие. Формула Байеса.} Одной из форм формулы полной вероятности является равенство: \( P(X_k|A) = \frac{P(A|X_k )P(X_k)}{P(A|X_1)P(X_1)+...+ P(A|X_n)P(X_n)} \)

\(\uparrow\) Для доказательства достаточно заметить, что: \(P(A \cap X_k)=P(A|X_k) \cdot P(X_k) = P(X_k|A) \cdot P(A)\)
Поэтому: \( P(X_K | A) =  \frac{P(A|X_k)P(X_k)}{P(A)}\)

Если \(X_1, X_2,..., X_n\) — попарно несовместные события, объединение которых совпадает со всем вероятностным пространством $U$, то в этом равенстве можно заменить $P(A)$ по формуле полной вероятности и получить искомую формулу. $\downarrow$

Формула Байеса истолковывается следующим образом: если существуют попарно исключающие друг друга гипотезы \(X_1, X_2,..., X_n\), охватывающие все возможные случаи, и если известны вероятности события $A$ при каждой из этих гипотез, то по формуле Байеса можно найти вероятность справедливости гипотезы $X_k$ при условии, что произошло событие $A$.
%слайд 29 ⬆️


\textbf{Пример 2.} Партия электрических лампочек на \(20 "percent"\) изготовлена заводом I, на 30 — заводом II и на 50 заводом III.

Для завода I вероятность выпуска бракованной лампочки равна 0,01, для завода II ——— 0,005, для завода III ——— 0,006. Какова вероятность, что взятая наугад лампочка оказалась бракованной?

Нас интересует событие $A$ — взятая наугад лампочка оказалась бракованной.

Рассмотрим три события: \begin{enumerate}
    \item X1 — взятая лампочка изготовлена I заводом,
    \item X2 — взятая лампочка изготовлена II заводом,
    \item X3 — взятая лампочка изготовлена III заводом.
\end{enumerate}

Эти три события несовместны, их объединение составляет всё пространство
возможных исходов, P(X1)=0,2, P(X2)=0,3, P(X3)=0,5.

Кроме того P(A|X1)=0,01, P(A|X2)=0,005, P(A|X3)=0,006.

Тогда по формуле полной вероятности: \(P(A)=P(A|X_1)P(X_1)+P(A|X_2)P(X_2)+P(A|X_3)P(X_3)= =0,01 \cdot 0,2 + 0,005 \cdot 0,3 + 0,006 \cdot 0,5 = 0,002 + 0,0015 + 0,003 = 0,0065.\)

\textbf{Замечание.} Формула полной вероятности сохраняется, если вместо условия U=X1ÈX2È...ÈXn, выполняется условие AÌX1ÈX2È...ÈXn.
%доделат слайд 30 ⬆️


\textbf{Пример 3.} В цехе стоят $a$ ящиков с исправными деталями и $b$ ящиков с бракованными деталями. Среди исправных деталей $p$ процентов отникелированы, а из числа бракованных никелированы лишь $q$ процентов (в каждом ящике). Вынутая наугад деталь оказалась никелированной. Какова вероятность, что она исправна?

Обозначим события: $X_1$ ——— деталь исправна, $X_2$ ——— деталь с браком, $A$ ——— деталь отникелирована.

Интересует \(P(X_1|A)\).

Имеем: \( P(X_1) = \frac{a}{a + b}, \ P(X_2) = \frac{b}{a + b},  \ P(A|X_1) = \frac{p}{100},  \ P(A|X_2) = \frac{q}{100}\)

Подставляя в формулу Байеса: \( P(X_1|A) = \frac{P(A|X_1)P(X_1)}{P(A)} = \frac{P(A|X_1)P(X_1)}{P(A|X_1)P(X_1)+P(A|X_2)P(X_2)} = \frac{ \frac{p}{100} \frac{a}{a + b} }{ \frac{p}{100} \frac{a}{a + b} + \frac{q}{100} \frac{b}{a + b}} = \frac{pa}{pa + qb}\).

%%dodelatь формулу байеса

%%Если a=50, b=3, p=90, q=5, то P(X1|A)=(50×90)/(50×90+3×5)=0,9967, если a=b=50, p=75, q=15, то P(X1|A)=(50×75)/(50×75+50×15)=75/(75+15)= =5/(5+1)=0,8333...
%доделат слайд 31 ⬆️


\subsection{Геометрическая вероятность}
\textbf{Пример 1}  Стержень наудачу разламывается на три части. Какова вероятность, что из получившихся отрезков можно будет получить треугольник?


\end{document}