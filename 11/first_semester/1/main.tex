\documentclass{article}

\usepackage[utf8x]{inputenc}
\usepackage[english,russian]{babel}
\usepackage{cmap}
\usepackage{commath}
\usepackage{amsmath}
\usepackage{amsfonts}
\usepackage{mathtools}
\usepackage{amssymb}
\usepackage{parskip}
\usepackage{titling}
\usepackage{color}
\usepackage{hyperref}
\usepackage{cancel}
\usepackage{enumerate}
\usepackage{multicol}
\usepackage{graphicx}
\usepackage[font=small,labelfont=bf]{caption}
\usepackage[a4paper, left=2.5cm, right=1.5cm, top=2.5cm, bottom=2.5cm]{geometry}

\graphicspath{ {./images/} }
\setlength{\droptitle}{-3cm}
\hypersetup{ colorlinks=true, linktoc=all, linkcolor=blue }
\pagenumbering{arabic}

\begin{document}
  \addtocontents{toc}{\protect\contentsline{section}{\protect\numberline{}Первый семестр}{}{}}
  \section{Пределы функций}
  \subsection{Теорема Вейерштрасса о предельной точке}

  \(\mathbb{E}\) --- числовое множество.

  \textbf{Определение 1.} Число \(a\) называется предельной точкой числового множества \(\mathbb{E}\), если \(\ \forall \varepsilon > 0\) в \\
  \(\varepsilon\)-окрестности точки \(a(U_\varepsilon(a))\ \exists\) хотя бы одна точка \(x\) из \(\mathbb{E};\ x \neq a\)

  \textbf{Определение 2.} Число \(a\) называется предельной точкой числового множества \(\mathbb{E}\),
  если в \(\ \forall U_\varepsilon(a)\) содержится \(\infty\) число элементов из \(\mathbb{E}\).

  \textbf{Теорема.} Если \(a\) --- предельная точка ч.м. \(\mathbb{E}\), то в \(\mathbb{E}\ \exists \{x_n\}:\ x_n \neq x_m, x_n \neq a\ \forall n:\ x_n \xrightarrow[n \rightarrow \infty]{} a\)

  \(\uparrow\) по определению 1

  \begin{enumerate}
      \item \(\varepsilon_{1} > 0 \xRightarrow[]{\textrm{Опр. 1}}\) в \(U_{\varepsilon_{1}}(a)\ \exists x_1:\ \begin{cases}x_1 \in \mathbb{E}\\ x_1 \neq a \end{cases}\) и \(\abs{x_1 - a} < \varepsilon_1\)
      \item \(\varepsilon_{2} \leq \frac{\abs{x_1 - a}}{2} \xRightarrow[]{\textrm{Опр. 1}}\) в \(U_{\varepsilon_{2}}(a)\ \exists x_2:\ \begin{cases}x_2 \in \mathbb{E}\\ x_2 \neq a \end{cases}\) и \(\abs{x_2 - a} < \varepsilon_2\)
      \item \(\varepsilon_3 \leq \frac{\abs{x_2-a}}{2} \xRightarrow[]{\textrm{Опр. 1}}\) в \(U_{\varepsilon_3}(a)\ \exists x_3:\ \begin{cases}x_3 \in \mathbb{E}\\ x_3 \neq a \end{cases}\) и \(\abs{x_3-a} < \varepsilon_3\)
      \item[$\vdots\;\;$]
      \item[$n$.] \(\varepsilon_{n} (> 0) \leq \frac{\abs{x_{n-1}-a}}{2} \xRightarrow[]{\textrm{Опр. 1}}\ \exists x_n: \begin{cases}x_n \in \mathbb{E}\\ x_n \neq a \end{cases}\) и \(\abs{x_n - a} < \varepsilon_{n}\)
      \item[$\vdots\;\;$]
  \end{enumerate}

  \(\Rightarrow \{x_n\}\)

  Осталось \(\{x_n\} \xrightarrow[]{\textrm{?}} a\)

  \(\ \forall\ \varepsilon > 0\ \exists N:\ \forall n > N: \abs{x_n - a} < \varepsilon\)

  \(\abs{x_n - a} < \varepsilon_n \leq \frac{\abs{x_{n-1}-a}}{2} < \frac{\varepsilon_{n-1}}{2} \leq \frac{\abs{x_{n-2}-a}}{2^2} < \frac{\varepsilon_{n-2}}{2^2} < ... < \frac{\varepsilon_{1}}{2^{n-1}} < \frac{\varepsilon_1}{n} < \varepsilon\)

  \(n > \frac{\varepsilon_1}{\varepsilon};\ N=[\frac{\varepsilon_1}{\varepsilon}] + 1\)
  \(\downarrow\)

  \textbf{Замечание.} Верно и утверждение обратное к утверждению теоремы.

  \textbf{Определение 3.} \(a\) --- предельная точка числового множества \(\mathbb{E} \Leftrightarrow\) когда в \(\mathbb{E}\ \exists\) последовательность различных \(x_n: x_n \neq a\ \forall n\) и \(x_n \xrightarrow[n \rightarrow \infty]{} a\).

  Опр. 1 \(\Leftrightarrow\) Опр. 2 \(\Leftrightarrow\) Опр. 3

  \(a \in \mathbb{E}\) или \(a \not\in \mathbb{E}\)

  \begin{enumerate}
      \item \(\mathbb{E} = \{x=\frac{1}{n};\ n \in N\}\)

      \(a = 0\) --- предельная точка \(\mathbb{E}, a \not \in \mathbb{E}\)

      \item \(\mathbb{Q};\ \forall_q\) --- предельная точка

      \item \(\mathbb{N}\) нет предельных точек

      \item M --- конечное. Нет предельных точек
  \end{enumerate}

  \subsubsection{Теорема Вейерштрасса о предельной точке}
  \textbf{Теорема.} Если бесконечное числовое множество \(\mathbb{E}\) ограничено, то для него существует хотя бы одна предельная точка.

  \(\uparrow\) Т.к. \(\mathbb{E}\) ограничено, \(\ \exists [m; M]:\ \mathbb{E} \subset  [m; M]\)

  \begin{enumerate}
      \item Делим \([m; M]:\ \begin{cases}[m; \frac{m+M}{2}]\\ [\frac{m+M}{2}; M]\end{cases}\)
      и выбираем ту часть, в которой содержится \(\infty\) число эл-тов из \(\mathbb{E}\), в ней берём какой-то \(x_1\)
      \item Снова делим на 2 и в одной из частей берем \(x_2 \neq x_1\)
  \end{enumerate}
  и т.д. \(\Rightarrow \{x_n\} \in \mathbb{E}\)

  по теореме Больцано-Вейерштрасса(или вложенных промежутков) \(x_n\) -- сх-ся к \(a\).

  Это \(a\) --- предельная точка \(\mathbb{E}\), по Опр.3.
  \(\downarrow\)

  \subsection{Предел функции в точке}

  \(y=f(x):\ X \rightarrow Y, a\) --- предельная точка \(X:\ \lim_{x \rightarrow a}{f(x)} = A\)

  \textbf{Определение 1.} Будем говорить, что существует предел функции f(x) при \(x \rightarrow a\) \( (lim_{x\rightarrow a} f(x) = A) \), если

  \begin{enumerate}
      \item функция \(f(x)\) определена в некоторой окрестности точки \(x=a\); за исключением, возможно, самой \(x=a\)
      \item \(\ \forall \varepsilon > 0\ \exists \delta(\varepsilon):\ \forall x: 0 < \abs{x - a} < \delta\) выполняется \(\abs{f(x) - A} < \varepsilon\).
  \end{enumerate}
  \(a - \delta < x < a + \delta\ x \neq a\).

  \(A - \varepsilon < f(x) < A + \varepsilon\)

  \textbf{Отрицание.} \(\ \exists \varepsilon > 0:\ \forall \delta(\varepsilon)\ \exists x: 0 < \abs{x - a} < \delta\) и \(\abs{f(x) - A} \geq \varepsilon\)

  \textbf{Определение 2.} Будем говорить, что существует предел функции f(x) при \(x \rightarrow a\) \( (lim_{x\rightarrow a} f(x) = A \textrm{ или } f(x) \xrightarrow[x \rightarrow a]{} A)\); если

  \begin{enumerate}
    \item \(f(x)\) определено в \(U_\delta(a)\)
    \item \(\forall x_n \rightarrow a \Rightarrow y_n = f(x_n)\rightarrow A\)
  \end{enumerate}
  (Определение по Гейне)

  Доказательство Опр. 1 \(\Leftrightarrow\) Опр. 2:
  \(\uparrow\)
  \begin{enumerate}
    \item ``\(\Rightarrow\)'' из Опр. 1 \(\Rightarrow\) Опр. 2\\
    надо, что если \(\forall x_n \rightarrow a \Rightarrow y_n = f(x_n) \rightarrow A\)

    ? (\(\forall \varepsilon > 0\ \exists N:\ \forall n > N \abs{f(x) - A} < \varepsilon\))

    \(\forall \varepsilon > 0 \xRightarrow[]{\textrm{Опр.1}}\ \exists \delta(\varepsilon):\) по \(\delta(\varepsilon)\) т.к. \(x_n \rightarrow a\).\\
    \(\ \exists N:\ \forall n > N\ \abs{x_n-a}<\delta \xRightarrow[]{\textrm{Опр.1}} \abs{f(x_n)-A}<\varepsilon\)

    \item ``\(\Leftarrow\)'' дано \(\forall x_n \rightarrow a\ f(x) \rightarrow A\).\\ От противного. Пусть Опр.1 не выполняется, т.е.
    \(\ \exists \varepsilon > 0:\ \forall \delta\ \exists x:\ 0 < \abs{x-a}<\delta\), но \(\abs{f(x)-A}\geq \varepsilon\).

    Рассмотрим \(\delta_k = \frac{1}{k}, k \in \mathbb{N},\) тогда \(\ \exists x_k:\ 0<\abs{x_k-a}<\delta=\frac{1}{k}\)
    \begin{equation*}
      a-\frac{1}{k}<x_k<a+\frac{1}{k}(*)
    \end{equation*}

    \(x_k\) --- сх-ся, \(f(x_k)\) (*) --- расх-ся (против Опр. 2)
    \(\downarrow\)
  \end{enumerate}
\end{document}