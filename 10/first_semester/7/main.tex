\documentclass{article}
% PKGS START
\usepackage[utf8x]{inputenc}
\usepackage[english,russian]{babel}
\usepackage{cmap}
\usepackage{commath}
\usepackage{amsmath}
\usepackage{amsfonts}
\usepackage{mathtools}
\usepackage{amssymb} 
\usepackage{parskip}
\usepackage{titling}
\usepackage{color}
\usepackage{hyperref}
\usepackage{cancel}
\usepackage{enumerate}
\usepackage{graphicx}
\usepackage{multicol}
\usepackage[a4paper, left=2.5cm, right=1.5cm, top=2.5cm, bottom=2.5cm]{geometry}
% PKGS END
% INIT START
\graphicspath{ {./images/} }
\setlength{\droptitle}{-3cm}
\hypersetup{
    colorlinks=true, %set true if you want colored links
    linktoc=all,     %set to all if you want both sections and subsections linked
    linkcolor=blue,  %choose some color if you want links to stand out
}

\pagenumbering{arabic}
% INIT END
\begin{document}
    \section{Элементы логики}

    \subsection{Высказывания. Алгебра высказываний}

    Под высказыванием понимают всякое утверждение, про которое имеет смысл говорить, что оно \textit{истинно} или \textit{ложно}.

    \textbf{Примеры.}
    \begin{enumerate}
        \item ``7 --- простое число'' --- высказывание.
        \item ``Существуют внеземные цивилизации'' --- высказывание.
        \item ``\(x>2\)'' --- не является высказыванием.
    \end{enumerate}
    Также не являются высказываниями определения, призывы, вопросы и т.д.

    \textbf{Обозначения.} Для обозначения высказываний будем использовать заглавные буквы латинского алфавита \(A, B, C, ...\)\\
    Например: \(A=\{5>6\}, B=\{\textrm{7 --- простое число}\}\).

    С помощью \textit{логических связок} можно образовывать новые высказывания.

    \subsubsection{Сумма высказываний}

    \textbf{Определение.} Высказывание, составленное из данных высказываний \(A\) и \(B\) при помощи союза ``или'' называют \textit{суммой(дизъюнкцией)} высказываний и обозначают \(A \vee B\). \(A \vee B\) истинно т.т.т., когда истинно хотя бы одно из высказываний его составляющих.
    
    \begin{center}
        \begin{tabular}{||c | c | c||}
            \hline
            \(A\) &  \(B\) & \(A \vee B\)\\
            \hline \hline
            И & И & И\\
            \hline
            И & Л & И\\
            \hline
            Л & И & И\\
            \hline
            Л & Л & Л\\
            \hline
        \end{tabular}
    \end{center}

    \subsubsection{Произведение высказываний}

    \textbf{Определение.} Высказывание, составленное из данных высказываний \(A\) и \(B\) при помощи союза ``и'' называют \textit{произведением(конъюнкцией)} высказываний и обозначают \(A \wedge B\). \(A \wedge B\) истинно т.т.т., когда истинны оба высказывания, его составляющие.

    \begin{center}
        \begin{tabular}{||c | c | c||}
            \hline
            \(A\) &  \(B\) & \(A \wedge B\)\\
            \hline \hline
            И & И & И\\
            \hline
            И & Л & Л\\
            \hline
            Л & И & Л\\
            \hline
            Л & Л & Л\\
            \hline
        \end{tabular}
    \end{center}

    \subsubsection{Импликация высказываний}

    \textbf{Определение.} Высказывание, составленное из данных высказываний \(A\) и \(B\) при помощи слов ``если ..., то ...'' называют \textit{импликацией} высказываний и обозначают \(A \rightarrow B\). Высказывание \(A\) называется \textit{условием} или \textit{посылкой}, а высказывание \(B\) --- \textit{заключением}. Импликация \(A \rightarrow B\) считается ложным высказыванием только в том случае, если посылка истинна, а заключение --- ложно.

    \begin{center}
        \begin{tabular}{||c | c | c||}
            \hline
            \(A\) &  \(B\) & \(A \rightarrow B\)\\
            \hline \hline
            И & И & И\\
            \hline
            И & Л & Л\\
            \hline
            Л & И & И\\
            \hline
            Л & Л & И\\
            \hline
        \end{tabular}
    \end{center}

    \subsubsection{Отрицание высказывания}

    \textbf{Определение.} \textit{Отрицанием} высказывания \(A\) называют высказывание \(\overline{A}\), заключающееся в том, что высказывание \(A\) --- ложно.

    \begin{center}
        \begin{tabular}{||c | c||}
            \hline
            \(A\) & \(\overline{A}\)\\
            \hline \hline
            И & Л\\
            \hline
            Л & И\\
            \hline
        \end{tabular}
    \end{center}

    \subsubsection{Алгебра высказываний}

    \textbf{Пример.}

    \begin{tabular}{||c | c | c | c | c||}
        \hline
        \(A\) &  \(B\) & \(\overline{A}\) & \(\overline{B}\) & \(\overline{B} \rightarrow \overline{A}\)\\
        \hline \hline
        И & И & Л & Л & И\\
        \hline
        И & Л & Л & И & Л\\
        \hline
        Л & И & И & Л & И\\
        \hline
        Л & Л & И & И & И\\
        \hline
    \end{tabular} \qquad
    \begin{tabular}{||c | c | c||}
        \hline
        \(A\) &  \(B\) & \(A \rightarrow B\)\\
        \hline \hline
        И & И & И\\
        \hline
        И & Л & Л\\
        \hline
        Л & И & И\\
        \hline
        Л & Л & И\\
        \hline
    \end{tabular}

    Т.о. каковы бы ни были высказывания \(A\) и \(B\), сложные высказывания \(\overline{B} \rightarrow \overline{A}\) и \(A \rightarrow B\) истинны и ложны одновременно.

    Такие высказывания называются \textit{равносильными}, (\(\overline{B} \rightarrow \overline{A} = A \rightarrow B\)).

    \textbf{Определение.} Тождественно истинным(\(I\)) называется высказывание, которое истинно не зависимо от истинности или ложности высказываний, его составляющих.\\
    Например, \(A \vee \overline{A} = I\).

    \textbf{Определение.} Тождественно ложным(\(L\)) называется высказывание, которое ложно не зависимо от истинности или ложности высказываний, его составляющих.\\
    Например, \((B \wedge A) \wedge \overline{A} = L\).

    \begin{enumerate}
        \item \(A \vee B = B \vee A\).
        \item \(A \wedge B = B \wedge A\).
        \item \(A \vee (B \vee C) = (A \vee B) \vee C\).
        \item \(A \wedge (B \wedge C) = (A \wedge B) \wedge C\).
        \item \(A \wedge (B \vee C) = (A \wedge B) \vee (A \wedge C)\).
        \item \(A \vee (B \wedge C) = (A \vee B) \wedge (A \vee C)\).
        \item \(\overline{A \vee B} = \overline{A} \wedge \overline{B}\).
        \item \(\overline{A \wedge B} = \overline{A} \vee \overline{B}\).
        \item \(\overline{\overline A} = A; A \vee A = A; A \wedge A = A\).
        \item \(A \vee \overline{A} = I; A \wedge \overline{A} = L\).
        \item \(A \vee I = I; A \wedge I = A\).
        \item \(A \vee L = L; A \wedge L = L\).
        \item \(A \rightarrow B = \overline{A} \vee B\).
    \end{enumerate}

    \textbf{Пример.} Известно, что на вопрос ``Кто из трёх учеников отличник?'' получен верный ответ: ``Если Андрей --- отличник, то и Вася --- отличник, но не верно, что если Борис --- отличник, то и Вася --- отличник''. Кто отличник?

    Рассмотрим три высказывания: \(A=\{\textrm{Андрей --- отличник}\}, B=\{\textrm{Борис --- отличник}\}, V=\{\textrm{Вася --- отличник}\}\).

    Запишем с помощью логических связок полученный ответ: \((A \rightarrow V) \wedge (\overline{B \rightarrow V})\).

    Этот ответ верный. Преобразуем выражение по правилам алгебра.

    \((A \rightarrow V) \wedge (\overline{B \rightarrow V}) = (\overline{A} \vee V) \wedge (\overline{\overline B \vee V}) = (\overline{A} \vee V) \wedge (\overline{\overline B} \wedge \overline{V}) = (\overline{A} \vee V) \wedge B \wedge \overline{V}) = (\overline{A} \vee V) \wedge \overline{V}) \wedge B = [(\overline{A} \wedge \overline{V}) \vee (V \wedge \overline{V})] \wedge B = [(\overline{A} \wedge \overline{V}) \vee L] \wedge B = \overline{A} \wedge \overline{V} \wedge B\)

    Произведение истинно, когда истинно каждое высказывание, значит, Борис --- отличник.

    \subsection{Предложения, зависящие от переменной}

    В логике утверждения(предложения), зависящие от одной или нескольких переменных, называются \textit{предикатами}.

    \textbf{Пример.} ``\(n\) --- простое число'' \(n \in \mathbb N\). При \(n = 2, 3, 5\) --- истина, при \(n = 4, 6, 18\) --- ложь.

    Будем обозначать \(A(n),\ B(x),\ C(x,y)\) и т.д.
    Для каждого такого предложения должно быть указано, на каком множестве оно определено(например, \(A(x),\ x \in \mathbb{R}\)).
    
    Итак, \(A(x), x \in U\) не является высказыванием на всём \(U\), но если \(A(x)\) рассмотреть при некоторых конкретных значениях \(x = x_0\), то \(A(x_0)\) будет либо истинно, либо ложно, т.е. будет высказыванием.
    
    \begin{center}
        \(U:\)
    \end{center}
    \begin{multicols}{2}
        \begin{center}
            \(x: A(x)\) --- истинно.
        \end{center}
        Это множество называется \textit{множеством истинности} предложения \(A(x)\).
        \begin{center}
            A
        \end{center}
        \textbf{Пример.} \(A(x)=\{x^2-x<0\}\)
        \columnbreak

        \begin{center}
            \(x: A(x)\) --- ложно.
        \end{center}
        \begin{center}
            \(\overline{A} = U\backslash A\)
        \end{center}
        \textbf{Пример.} \(A = (0,1), \quad \overline{A} = (-\infty, 0] \cup [1, +\infty)\)
    \end{multicols}

    \textbf{Определение.} Два предложения \(A(x)\) и \(B(x)\), заданные на одном и том же множестве, называются равносильными, если их множества истинности совпадают.

    \subsubsection{Логические операции}

    \textbf{Определение.} \textit{Отрицанием} предложения \(A(x),\ x \in U\), называется предложение, определённое на том же множестве \(U\) и обращающееся в истинное высказывание для т.т.т. значений \(x\), для которых \(A(x)\) --- ложно. Обозначается \(\overline{A(x)}\).

    \textbf{Определение.} \textit{Импликацией} \(A(x) \rightarrow B(x)\) предложений \(A(x)\) и \(B(x)\), определённых на множестве \(U\), называется предложение, определённое на том же множестве и обращающееся в ложное высказывание для т.т.т. значений \(x\), для которых \(A(x)\) --- истинно, а \(B(x)\) --- ложно.

    Аналогично можно дать определения \textit{суммы} и \textit{произведения} предложений \(A(x)\) и \(B(x)\).

    \textbf{Пример.} \(A(x)=\{x-2>0\}, B(x)=\{x+2\geq 0\}\).
    
    \begin{enumerate}
        \item \(A(x) \vee B(x) = \{x-2>0\ \textrm{или}\ x+2\geq 0\}\). Множество истиности \([-2, +\infty)\).
        \item \(A(x) \wedge B(x) = \{x-2>0\ \textrm{и}\ x+2\geq 0\}\). Множество истиности \((2, +\infty)\).
        \item \(A(x) \rightarrow B(x) = \{\textrm{Если}\ x-2>0,\ \textrm{то}\ x+2\geq 0\}\). Ложно, если \(x-2>0\) и \(x+2<0\), \(x>2\) и \(x<-2\), \(\emptyset\). Т.е. множество истиности \(\mathbb{R}\).
        
        \(A(x) \rightarrow B(x) \Leftrightarrow \overline{A(x)} \vee B(x)\).
    \end{enumerate}

    С предложениями, зависящими от переменной, связаны два вида часто встречающихся утверждений:

    \begin{enumerate}
        \item Предложение \(A(x)\) обращается в истинное высказывание для всех элементов множества \(U\)
        \begin{center}
            \(\forall x \in U\ A(x)\).
        \end{center}
        \item Предложение \(A(x),\ x \in U\) обращается в истинное высказывание хотя бы для одного элемента множества \(U\)
        \begin{center}
            \(\exists\ x \in U: A(x)\).
        \end{center}
    \end{enumerate}

    каждое из них --- высказывание.

    \textbf{Пример.}
    
    \begin{enumerate}
        \item Рассмотрим предложение \(A(x) = \{x^2 > 0,\ x \in \mathbb R\}\).
        
        Высказывание \(\forall x \in \mathbb{R}\ x^2 > 0\) --- ложно, а высказывание \(\exists x \in \mathbb{R}: x^2 > 0\) --- истинно.
        \item \(B(x) = \{x^2 \geq 0,\ x \in \mathbb R\}\).
        
        Высказывание \(\forall x \in \mathbb{R}\ x^2 \geq 0\) --- истинно, и высказывание \(\exists x \in \mathbb{R}: x^2 \geq 0\) --- истинно.
    \end{enumerate}
\end{document}