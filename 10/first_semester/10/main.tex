\documentclass{article}
% PKGS START
\usepackage[utf8x]{inputenc}
\usepackage[english,russian]{babel}
\usepackage{cmap}
\usepackage{commath}
\usepackage{amsmath}
\usepackage{amsfonts}
\usepackage{mathtools}
\usepackage{amssymb} 
\usepackage{parskip}
\usepackage{titling}
\usepackage{color}
\usepackage{hyperref}
\usepackage{cancel}
\usepackage{enumerate}
\usepackage{graphicx}
\usepackage[a4paper, left=2.5cm, right=1.5cm, top=2.5cm, bottom=2.5cm]{geometry}
% PKGS END
% INIT START
\graphicspath{ {./images/} }
\setlength{\droptitle}{-3cm}
\hypersetup{
    colorlinks=true, %set true if you want colored links
    linktoc=all,     %set to all if you want both sections and subsections linked
    linkcolor=blue,  %choose some color if you want links to stand out
}

\pagenumbering{arabic}
% INIT END
\begin{document}
    \section{Элементы комбинаторики}
    Рассматриваем множество \(X\), содержащее \(n\) различных элементов.
    \subsection{Размещения без повторений}
    \textbf{Определение 1.} Упорядоченное подмножество, состоящее из \(k\) элементов \(n\)-элементного множества \(X\), называют размещением без повторений из \(n\) элементов по \(k\) элементов. 

    \(\Rightarrow n \ge k \ge 0\) и размещения из \(n\) по \(k\) элементов - это все \(k\)-элементные подмножества множества \(X\), отличающиеся друг от друга порядком следования элементов.

    \textbf{Пример.} Для множества \(X = \{a, b, c\}\) его размещения из 3-х по 2 элементва выглядят так:
    \\\(\{a, b\}, \{a, c\}, \{b, c\}, \{b, a\}, \{c, a\}, \{c, b\}.\) Их 6 штук.

    \(A_n^k\) - число размещений без повторений из \(n\) элементов по \(k\) элементов.

    \textbf{Задача.} Найти число размещений без повторений из \(n\) элементов по \(k\) элементов.

    \(\uparrow\) 1) Найдем \(A_n^1\). Элементом подмножества \((x_1)\) может быть любой из элементов множества \(\Rightarrow\) имеем \(n\) возможностей выбора, т.е. \(A_n^1 = n\)

    2) Если первый элемент выбран, то второй можно выбрать лишь \(n-1\) способом. По правилу произведения выбор \((x_1,\ x_2)\) можно осуществить \(A_{n}^2 = n(n-1)\) способами.

    3) При каждом способе выбора первых двух элементов подмножетва \((x_1, x_2)\) выбор элемента из подмножества \((x_3)\) можно осуществить \(n-2\) способами \(\Rightarrow\) \(A_n^3 = n(n-1)(n-2)\) 
    
    и т.д....

    k) Выбор \(x_k\) можно осуществить \(n-k+1\) способами \(\Rightarrow\) \textbf{\(A_n^k = n(n-1)(n-2)*...*(n-k+1)\)}. \(\downarrow\)

    \textbf{Определение 2.} \(n!=1*2*3*...*n, 0!=1\).

    Тогда \(A_n^k = \frac{n(n-1)(n-2)*...*(n-k+1)(n-k)(n-k-1)*...*1}{(n-k)(n-k-1)*...*1} = \frac{n!}{(n-k)!}\)

    Отметим, что

    \(A_n^0 = \frac{n!}{n!} = 1\) (пустое множество можно выбрать из любого конечного множества одним способом)

    \(A_0^0 = \frac{0!}{0!} = 1\) (пустое выбрать из пустого - тоже одним способом)

    \textbf{Пример 1.} Сколькими способами можно опустить 5 писем в 11 почтовых ящиков, если в каждый ящик опускается не более одного письма?
    \\Для первого письма 11 возможностей для второго 10 и т.д. \(11*10*9*8*7=A_{11}^5=\frac{11!}{6!}\)

    \textbf{Пример 2.} Пусть множество \(X\) содержит \(k\) элементов, а множество \(Y\) - \(n\) элементов \((n \ge k)\). 
    Посчитаем число всевозможных обратимых отображеий множества \(X\) во множество \(Y\).

    Занумеруем элементы множества \(X = \{x_1, x_2, x_3, ..., x_k\}\)

    Под действием отображения \(f\ X = {x_1, x_2, x_3, ..., x_k} \rightarrow f(X) = {f(x_1), f(x_2), ..., f(x_k)}\).

    В качестве \(f(x_1)\) может выступать любой (но ровно один) из элементов множества \(Y\) (\(n\) вариантов). 
    При каждом варианте для \(f(x_1)\) в качестве \(f(x_2)\) - любой из оставшихся элементов множества \(Y\), так как отображение по условию обратимо (\(n-1\) вариант). 
    Т.е. для определения двух позиций позиций для \(f(X)\) имеем \(n(n-1)\) вариантов и т.д.
    Таким образом \(m = A_n^k\)

    \subsection{Перестановки без повторений}
    \textbf{Определение 1.} Перестановкой без повторений из \(n\) элементов называют размещение без повторений из этих элементов по \(n\).

    \textbf{Пример.} Для множества \(X = \{a, b, c\}\) выпишем все перестановки.
    \\\(\{a, b, c\}, \{b, c, a\}, \{c, a, b\}
    \\\{b, a, c\}, \{a, c, b\}, \{c, b, a\}\). Их 6 штук.

    Число перестановок обозначается \(P_n\).

    Для числа перестановок имеем \(P_n = A_n^n = \frac{n!}{0!} = n!\)

    \textbf{Пример 1.} Сколькими способами 6 человек могут сесть на 6 стульев, поставленных в ряд.

    Занумеруем стулья 1, 2, 3, 4, 5, 6 и обозначим человека, севшего на \(k\)-ый стул, через \(x_k\).
    Тогда \((x_1, x_2, ..., x_6)\) - перестановка из имен этих людей, причем каждой такой перестановке соответствует один и только один способ их размещения на стульях.
    \(m = P_6 = 6! = 1*2*3*4*5*6 = 120*6 = 720\).

    \textbf{Пример 2.} Отметим, что каждая перестановка элементов множества \(X\) задает взаимно однозначное отображение множества на себя.
    (Т.е. существует \(n!\) взаимно однозначных отображений конечного множества \(X\) на себя.)

    \subsection{Сочетания без повторений}
    \textbf{Определение 1.} \(k\)-элементное подмножество \(n\)-элементного множества \(X\) называют сочетанием без повторений из \(n\) элементов этого множества по \(k\).

    \textbf{Пример 1.} Для множества \(X = \{a, b, c\}\) выпишем все сочетания по 2, т.е. все 2-х элементные подмножества.
    \\\(\{a, b\}, \{b, c\}, \{a, c\}\). Их 3 шт.

    Число сочетаний обозначается \(C_n^k\)

    \textbf{Задача.} Найдем число сочетаний без повторений из \(n\) по \(k\) элементов.

    \(\uparrow\) Число сочетаний без повторений из \(n\) по \(k\) элементов равно \(C_n^k\)
    \\Заметим, что из каждого сочетания без повторений из \(n\) по \(k\) элементов путем упорядочивания можем получить \(k!\) различных размещений без повторений из \(n\) по \(k\) элементов.
    \\При этом различные сочетания порождают различные размещения и всякое размещение может быть получено указанным способом (путем упорядочивания некоторого сочетания).

    Таким образом, \(A_n^k = C_n^k*k! \Rightarrow C_n^k = \frac{A_n^k}{k!} = \frac{n!}{k!(n-k)!} \downarrow\)

    \textbf{Пример.} Сколькими способами можно составить команду из 4-х человек для участия в соревнованиях по бегу, если имеется 7 бегунов?

    Надо из 7 человек выбрать 4 без учета порядка \(\Rightarrow C_7^4 = \frac{7!}{4!3!} = \frac{5*6*7}{3!} = 35\)

    \textbf{Замечание.} Пусть в предыдущей задаче требуется выбрать команду из 4-х человек для эстафетного бега 100+200+400+800 м.

    Тогда число способов было бы \(A_7^4 = \frac{7!}{3!} = 4*5*6*7 = 840\),
    \\поскольку порядок спортсменов играл был роль.

    \textbf{Свойства чисел \(C_n^k\)}
    \begin{enumerate}
        \item \(C_n^k = C_n^{n-k}\) (правило симметрии)
        
        \(C_n^k = \frac{n!}{k!(n-k)!} = \frac{n!}{(n-k)!(n-(n-k))!} = C_n^{n-k}\)

        \item \(C_n^{k-1} + C_n^k = C_{n+1}^k\) (правило Паскаля)
        
        \(C_n^{k-1} + C_n^k = \frac{n!^{\backslash k}}{(k-1)!(n-k+1)!} + \frac{n!^{\backslash n-k+1}}{k!(n-k)!} = \frac{n!k+n!(n-k+1)}{k!(n-k+1)!} = \frac{n!(n+1)}{k!(n+1-k)!} = C_{n+1}^k\)
    \end{enumerate}

    \subsection{Размещения с повторениями}

    \textbf{Определение 1.} Любой упорядоченный набор из \(k\) элементов множества \(X\), содержащего элементы n \underline{видов}, называется размещением с повторениями.

    Обозначим число таких наборов через \(\overline{A_n^k}\)

    \textbf{Задача.} Покажем, что \(\overline{A_n^k} = n^k\)

    \(\uparrow\) Действительно, на первое место в наборе \((x_1, x_2, ..., x_k)\) можем взять любой из \(n\) видов.
    \\После выбора первого элемента на второе место - любой из элементов \(n\) видов \(\Rightarrow\) выбор первых двух элементов \((x_1, x_2)\) может быть осуществлен \(n*n=n^2\) способами (по правилу произведения).
    \\Далее, выбор трех элементов \((x_1, x_2, x_3)\) -- \(n^2*n=n^3\) способами и т.д. \(\Rightarrow \overline{A_n^k} = n^k \downarrow\)

    \textbf{Пример 1.} Сколько пятизначных чисел можно составить из цифр 1, 2, 3, 4, 5, 6, 7, 8, 9?
    \\\(9^5\)

    \textbf{Пример 2.} Найдите число всех отображений множества \(X\) в множество \(Y\), если \(m(X) = k, m(Y) = n\).

    Занумеруем элементы множества \(X = \{x_1, x_2, x_3, ..., x_k\}\).
    \\Под действием отображения \(f\ X=\{x_1, x_2, x_3, ..., x_k\} \rightarrow f(X) = \{f(x_1), f(x_2), ..., f(x_k)\}\).
    \\Сколько существует различных наборов \(\{f(x_1), f(x_2), ..., f(x_k)\}\)?

    Для \(f(x_1)\) существует \(n\) возможностей (любой из \(n\) элементов множества \(Y\)). Для \(f(x_2)\) -- \(n\) возможностей.
    Тогда для двух элементов \(n^2\) возможностей и т.д. Ответ: \(n^k\).

    \textbf{Пример 3.} Найти число всех подмножеств множества \(X\), если \(m(X) = k\) (включая \(\varnothing\) и само множество \(X\))?

    \textbf{Способ 1.} Можно так: \(C_k^0 + C_k^1 + ... + C_k^k\)

    \textbf{Способ 2.}
    \\Заметим, что всякое подмножество характеризуется тем, что \(x_i, i = 1, 2, ..., k\) принадлежит подмножеству или не принадлежит.

    Например, одноэлементное подмножество, состоящее только из \(x_1\): (+, -, -, ..., -), двухэлементное подмножество, состоящее из \(x_{k-1}\) и \(x_k\): (-, -, -, ..., +, +).

    И обратно, понятно, какое подмножество соответствует всякому набору из + и -.

    Таким образом, можем переформулировать вопрос: сколько существует упорядоченных наборов длины \(k\), содержащих элементы двух видов.

    Их \(2^k\).

    \subsection{Перестановки с повторениями}

    \textbf{Задача.} Сколько можно получить слов (возможно бессмысленных) при перестановке букв в слове <<tolpa>>? <<topot>>?

    Для слова <<tolpa>> ответ простой: 5!
    \\Если же переставлять буквы в слове <<topot>>, то число вариантов уменьшится.

    \textbf{Способ А.} \(t_1 o_2 p_3 o_4 t_5\). Имеем 5! перестановок занумерованных букв.

    Если я фиксирую все буквы, кроме <<t>>, то 
    \\<<\(t_1 o_2 p_3 o_4 t_5\)>> и <<\(t_5 o_2 p_3 o_4 t_1\)>>,
    \\<<\(t_1 p_3 o_2 o_4 t_5\)>> и <<\(t_5 p_3 o_2 o_4 t_1\)>> и т.д.
    \\это пары одинаковых слов.
    \\Для <<t>> имеется 2! перестановок для каждой перестановки занумерованных букв \(\Rightarrow\) надо 5! разделить на 2!.

    Кроме того, для каждой перестановки занумерованных букв 2! перестановки для букв <<o>>. Еще разделить на 2! Таким образом \(\frac{5!}{2!2!}\).

    \textbf{Способ Б.}

    Имеется 5 позиций. На них надо расставить 1 букву <<p>>, 2 буквы <<o>>, 2 буквы <<t>>.

    1 шаг: выбираю одну позицию для <<p>>: \(C_5^1\) способов. Осталось 4 позиции.

    2 шаг: выбираю 2 позиции для постановки двух букв <<o>>: \(C_4^2 \Rightarrow C_5^1C_4^2\) способов заполнения 3-х позиций.

    Осталось два места, которые заполним буквами <<t>>. \(C_5^1*C_4^2 = \frac{5!}{1!4!}*\frac{4!}{2!2!} = \frac{5!}{2!2!}\)

    В общем случае.

    \textbf{Определение 1.} Пусть имеются элементы \(n\) видов: \(k_1\) элементов 1-ого вида, \(k_2\) -- 2-го вида, \(k_3\) -- 3-го, ..., \(k_n\) -- n-ого. Любой упорядоченный набор из \(k\) элементов \(k = k_1 + k_2 + ... + k_n\) называется перестановкой с повторениями.

    Число таких перестановок обозначим \(P(k_1, k_2, ..., k_n)\)

    \textbf{Задача.} Найдем число перестановок \(P(k_1, k_2, ..., k_n)\) состава \((k_1, k_2, ..., k_n)\).
    
    \(\uparrow\) Будем действовать способом Б.

    1) Выбираем \(k_1\) место из \(k\) элементов 1-го вида. Это можно сделать \(C_{k_1+k_2+...+k_n}^{k_1}\) способами.

    2) Из оставшихся \(k - k_1\) мест выбираем \(k_2\) мест для элементов второго вида. По правилу произведения имеем \(C_{k_1+k_2+...+k_n}^{k_1}*C_{k_2+k_3+...+k_n}^{k_2}\) способов выбора для элементов 2-х видов.

    3) Для элементов 3-х видов: \(C_{k_1+k_2+...+k_n}^{k_1}*C_{k_2+k_3+...+k_n}^{k_2}*C_{k_3+k_4+...+k_n}^{k_3}\) 
    
    и т.д.

    n-1) \(C_{k_1+k_2+...+k_n}^{k_1}*C_{k_2+k_3+...+k_n}^{k_2}*...*C_{k_{n-1}+k_n}^{k_{n-1}}\)

    n) Элементы последнего вида расставим на оставшиеся места.

    \(P(k_1, k_2, ..., k_n) = \frac{(k_1 + k_2 + ... + k_n)!}{k_1!{\cancel{(k_2 + k_3 + ... + k_n)!}}}*\frac{\cancel{(k_2 + ... + k_n)!}}{k_2!(\cancel{k_3 + ... + k_n)!}}*\frac{\cancel{(k_3 + ... + k_n)!}}{k_3!(\cancel{k_4 + ... + k_n)!}}*...*\frac{\cancel{(k_{n-1} + k_n)!}}{k_{n-1}!k_n!}
    = \frac{(k_1 + k_2 + ... + k_n)!}{k_1!k_2!k_3!*...*k_n!} = \frac{k!}{k_1!k_2!k_3!*...*k_n!}\)

    \(P(k_1, k_2, ..., k_n) = \overline{P_k} = \frac{k!}{k_1!k_2!k_3!*...*k_n!}\)

    \subsection{Сочетания с повторениями.}

    \textbf{Задача.} Сколько наборов из 7 пирожных можно составить, если в продаже имеется 4 сорта пирожных?

    \(\uparrow\) А) Поймем, какого типа соединения нас интересуют. Так как состав наборов может отличаться, то это не перестановки, порядок не важен, значит не размещения \(\Rightarrow\) какие-то сочетания.

    Б) Решать можно, например, так:
    \\Каждый набор должен содержать 7 шт. пирожных: \(k_1\) шт. 1-го сорта, \(k_2\) шт. 2-го сорта, \(k_3\) шт. 3-го сорта, \(k_4\) шт. 4-го сорта -- \((k_1, k_2, k_3, k_4)\), где \(k_1 + k_2 + k_3 + k_4 = 7\). 
    Рассмотрим всевозможные такие наборы (скобки, кортежи).

    Каждому такому набору можем поставить в соответствие последовательность 0 и 1 следующим образом: 1 -- пирожное, 0 -- запятая. (Если количество пирожных какого-то сорта в наборе равно 0, то ничего не ставим.)

    (4, 0, 3, 0) \(\rightarrow\) (1111001110),
    \\(0, 2, 2, 3) \(\rightarrow\) (0110110111)
    \\всегда 7 единиц и 3 нуля!

    (1100111011) \(\rightarrow\) (2, 0, 3, 2)

    Таким образом, существует взаимно однозначное соответствие между наборами \((k_1, k_2, k_3, k_4)\) и последовательностями из 7-ми единиц и 3-х нулей.

    Тогда число различных наборов \((k_1, k_2, k_3, k_4)\) равно числу всевозможных последовательностей из 7-ми единиц и 3-х нулей.

    \(P(7, 3) = \frac{10!}{3!7!} = C_{7+4-1}^7 = 120 \downarrow\)
\end{document}