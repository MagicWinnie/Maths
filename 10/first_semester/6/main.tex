\documentclass{article}

\usepackage[utf8x]{inputenc}
\usepackage[english,russian]{babel}
\usepackage{cmap}
\usepackage{commath}
\usepackage{amsmath}
\usepackage{amsfonts}
\usepackage{mathtools}
\usepackage{amssymb}
\usepackage{parskip}
\usepackage{titling}
\usepackage{color}
\usepackage{hyperref}
\usepackage{cancel}
\usepackage{enumerate}
\usepackage{multicol}
\usepackage{graphicx}
\usepackage[font=small,labelfont=bf]{caption}
\usepackage[a4paper, left=2.5cm, right=1.5cm, top=2.5cm, bottom=2.5cm]{geometry}

\graphicspath{ {./images/} }
\setlength{\droptitle}{-3cm}
\hypersetup{ colorlinks=true, linktoc=all, linkcolor=blue }
\pagenumbering{arabic}

\begin{document}
    \subsection{Линейная функция и её график}

    Почти все прямые на плоскости могут быть описаны линейной функцией.

    \subsubsection{Уравнение прямой}

    Пусть точка \(A(x,y)\) принадлежит прямой \(l\), проходящей через точку \(A_0(x_0, y_0)\) и перпендикулярной вектору \(\overset{\rightarrow}{n} = (a_1, b_1)\). Выясним, какому уравнению удовлетворяют координаты точки \(A\).
    \begin{equation}\label{*}
        a_1x + b_1y = c_1,\quad a_1^2 + b_1^2 \not = 0
    \end{equation}
    Уравнение \ref{*} называется \textit{общим уравнением} прямой на плоскости.

    \textbf{Утверждение.} Любая прямая на плоскости задаётся уравнением \ref{*}.

    \(\uparrow\)
    \begin{enumerate}
        \item Мы уже показали, что если точка принадлежит некоторой прямой \(l\), проходящей через точку \(A_0(x_0, y_0)\)
        и перпендикулярной вектору \(\overset{\rightarrow}{n} = (a_1, b_1)\), то её координаты удовлетворяют уравнению типа \ref{*}.
        \item Покажем, что если координаты точки удовлетворяют уравнению \ref{*}, то она принадлежит прямой.
        
        Пусть \(b_1 \not = 0\). Рассмотрим точку \(B\) с координатами \((0, c_1/b_1)\). Запишем уравнение прямой, проходящей через эту точку \(B\) и перпендикулярной вектору \(\overset{\rightarrow}{n} = (a_1, b_1)\):

        \(a_1(x-0) + b_1(y - \frac{c_1}{b_1}) = 0\), т.е. \(a_1x+b_1y=c_1\).

        Таким образом, уравнение \ref{*} --- это уравнение некоторой прямой.\quad \(\downarrow\)
    \end{enumerate}

    \textbf{Пример.} \(x+y=2\). Это прямая, перпендикулярная вектору \(\overset{\rightarrow}{n} = (1, 1)\).

    \textbf{Утверждение.} Уравнение прямой, проходящей через точку \(A_0(x_0,y_0)\) и параллельной вектору \(\tau(\tau_1, \tau_2)\), имеет вид:
    \begin{equation}
        \frac{x - x_0}{\tau_1} = \frac{y - y_0}{\tau_2}
    \end{equation}
    Возьмём произвольную точку \(A(x, y) \in l\). Тогда вектор \(\overset{\longrightarrow}{A_0A} = (x - x_0, y - y_0)\). Так как \(\overset{\longrightarrow}{A_0A}\ ||\ \overset{\rightarrow}{\tau}\), то их координаты пропорциональны:
    \[\frac{x - x_0}{\tau_1} = \frac{y - y_0}{\tau_2}\]
    Уравнение прямой ещё можно записать в параметрическом виде:
    \begin{equation}
        \frac{x - x_0}{\tau_1} = \frac{y - y_0}{\tau_2} = \lambda,\ \lambda \in \mathbb{R}.
    \end{equation}
    \begin{center}
        \(\begin{cases}
            x = x_0 + \lambda \tau_1;\\
            y = y_0 + \lambda \tau_2.
        \end{cases}\), \(\lambda \in \mathbb{R}\).
    \end{center}

    \subsubsection{Система двух линейных уравнений с двумя неизвестными}

    \begin{center}
        \(\begin{cases}
            a_1x + b_1y = c_1\\
            a_2x + b_2y = c_2
        \end{cases}\)
    \end{center}
    Вам известны два подхода к решению таких систем:
    \begin{enumerate}
        \item Выразить из одного уравнения одно из неизвестных и подставить во второе.
        \item Сложить/вычесть уравнения системы с подходящими коэффициентами, чтобы избавиться от одного из неизвестных.
    \end{enumerate}
    Иногда, требуется не столько решить систему, сколько сказать: имеет ли эта система решения? Сколько их?

    Достаточно просто можно ответить на этот вопрос, если обратиться с геометрической интерпретации системы двух линейных уравнений с двумя неизвестными.

    Если \(a_1^2 + b_1^2 \not = 0\) и \(a_2^2 + b_2^2 \not = 0\), то каждое из уравнений системы мы можем рассматривать как уравнение прямой на плоскости.
    Значит, вопрос о числе решений системы двух линейных уравнений может быть рассмотрен как вопрос о взаимном расположении двух прямых на плоскости.

    Две прямые на плоскости могут:

    %images at 33:50

    \begin{multicols}{3}
        \(\exists!\) решение системы(точка)

        нормали не параллельны
        \[\frac{a_1}{a_2} \not = \frac{b_1}{b_2}\]
        \[\Delta = a_1b_2 - a_2b_1 \not = 0\]
        \columnbreak

        \(\infty\) число решений системы(прямая)\\
        нормали параллельны
        \[\frac{a_1}{a_2} = \frac{b_1}{b_2} = \frac{c_1}{c_2}\]
        \[\Delta = a_1b_2 - a_2b_1 = 0\]
        \[\Delta_1 = a_1c_2 - a_2c_1 = 0\]
        \[\Delta_2 = b_1c_2 - b_2c_1 = 0\]
        \columnbreak

        \(\emptyset\) решений системы

        нормали параллельны
        \[\frac{a_1}{a_2} = \frac{b_1}{b_2} \not = \frac{c_1}{c_2}\]
        \[\Delta = a_1b_2 - a_2b_1 = 0\]
        \[\Delta_1 = a_1c_2 - a_2c_1 \not = 0\]
        \[\Delta_2 = b_1c_2 - b_2c_1 \not = 0\]
    \end{multicols}

    Осталось рассмотреть случаи:
    \begin{enumerate}
        \item \(a_1^2 + b_1^2 \not = 0,\ a_2 = b_2 = 0\). Здесь, если \(c_2 = 0\), то система имеет бесконечно много решений(прямая), если \(c_2 \not = 0\), система не имеет решений;
        \item \(a_2^2 + b_2^2 \not = 0,\ a_1 = b_1 = 0\). Здесь, если \(c_1 = 0\), то система имеет бесконечно много решений(прямая), если \(c_1 \not = 0\), система не имеет решений;
        \item \(a_1 = b_1 = a_2 = b_2 = 0\). Здесь, если \(c_1 = c_2 = 0\), то система имеет бесконечно много решений(вся плоскость \(OXY\)), иначе, система не имеет решений.
    \end{enumerate}

    Пусть дана система \(\begin{cases} a_1x + b_1y = c_1\\ a_2x + b_2y = c_2 \end{cases}\)

    Введём обозначения:
    \[\Delta = a_1b_2 - a_2b_1\qquad \Delta_1 = a_1c_2 - a_2c_1\qquad \Delta_2 = b_1c_2 - b_2c_1\]

    \(\Delta = 0 \overset{\textrm{нет}}{\longrightarrow}\) ед. решение, совместная система, точка

    \(\downarrow\)да

    \(\Delta_1 = \Delta_2 = 0 \overset{\textrm{нет}}{\longrightarrow} \emptyset\) решений, несовместная система, \(||\) прямые

    \(\downarrow\)да

    \(a_1 = b_1 = a_2 = b_2 = 0 \overset{\textrm{нет}}{\longrightarrow} \infty\) решений, неопределённая система, прямая

    \(\downarrow\)да

    \(c_1 = c_2 = 0 \overset{\textrm{нет}}{\longrightarrow} \emptyset\) решений, несовместная система

    \(\downarrow\)да

    \(\infty\) решений, неопределённая система, плоскость

    \subsection{Квадратичная функция и её график}

    Рассмотрим квадратичную функцию: \(y = ax^2+bx+c,\ a \not = 0\).

    С помощью каких преобразований получается \(y=ax^2+bx+c\) из \(y=x^2\)?

    \(y = a(x^2+\frac{b}{a}x+\frac{c}{a})=a((x+\frac{b}{2a})^2 - \frac{b^2}{4a^2}+\frac{c}{a})=a((x+\frac{b}{2a})^2 - \frac{b^2-4ac}{4a^2})=
    a(x+\frac{b}{2a})^2-\frac{b^2-4ac}{4a}=a(x+\frac{b}{2a})^2 - \frac{D}{4a}\)

    О квадратичной функции общего вида, в зависимости от значений коэффициентов, мы можем сказать следующее:

    \begin{enumerate}
        \item Симметрия относительно \(OX\): если \(a>0\), то ветви параболы направлены вверх, если \(a<0\), то ветви параболы направлены вниз. Растяжение/сжатие в \(a\) раз вдоль оси \(OY\), т.е. если \(|a|>1\), то парабола более узкая, если \(|a|<1\), то парабола более широкая.
        \item Параллельный перенос на вектор \((-\frac{b}{2a}, -\frac{D}{4a})\).
    \end{enumerate}

    \subsubsection{Свойства квадратичной функции}

    \begin{enumerate}
        \item Если \(a>0\), то ветви параболы направлены вверх, если \(a<0\), то ветви параболы направлены вниз;
        \item Все параболы с одинаковым коэффициентом \(a\) получаются друг из друга параллельным переносом;
        \item Вершина параболы имеет координаты \((-\frac{b}{2a},-\frac{D}{4a})\);
        \item Появление корней.
    
        Смотрим на выражение, отвечающее за сдвиг по оси \(OY\): \(-\frac{D}{4a}\).
        %images at 1:02:57
        \item Парабола общего вида симметрична относительно прямой \(x=-\frac{b}{2a} \Rightarrow\) корни симметричны относительно \(x=-\frac{b}{2a}\).
        \item Формула корней уравнения \(ax^2+bx+c=0\):
        
        \(ax^2+bx+c=a(x+\frac{b}{2a})^2-\frac{D}{4a}=a((x+\frac{b}{2a})^2-\frac{D}{4a^2})=a(x+\frac{b}{2a}-\frac{\sqrt{D}}{2a})(x+\frac{b}{2a}+\frac{\sqrt{D}}{2a})=a(x-x_1)(x-x_2)\)
        \begin{equation*}
            x_{1,2}=\frac{-b\pm\sqrt{D}}{2a}
        \end{equation*}

        Полезная формула \(|x_2-x_1|=\frac{\sqrt{D}}{|a|}\).
        \item Чтобы задать квадратичную функцию, надо задать три числа \(a, b, c\) или какие-либо 3 условия(например, три точки).
        При этом получим линейную относительно \(a, b, c\) систему трёх уравнений на три неизвестные. Также как и система двух линейных уравнений на две неизвестные, эта система может иметь единственное решение, может не иметь решений, может иметь бесконечно много решений.
        \item Формулы Виета
        
        \(a(x-x_1)(x-x_2)=a(x^2-(x_1+x_2)x+x_1x_2)=ax^2-a(x_1+x_2)x+ax_1x_2\)

        Две параболы совпадают \(\Leftrightarrow\) когда совпадают их коэффициенты, значит
        \begin{equation*}
            \begin{cases} b = -a(x_1+x_2)\\ c = ax_1x_2 \end{cases} \qquad \begin{cases} x_1+x_2=-\frac{b}{a}\\ x_1x_2=\frac{c}{a} \end{cases}
        \end{equation*}
        Таким образом, если квадратное уравнение имеет корни(D > 0), то
        \begin{equation*}
            \begin{cases}
                x_1+x_2=-\frac{b}{a}\\
                x_1x_2=\frac{c}{a}
            \end{cases}
        \end{equation*}
        
    \end{enumerate}

\end{document}